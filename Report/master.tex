%\documentclass[11pt]{report}
%\usepackage[a4paper]{geometry}

\documentclass[ % the name of the author
                    author={Aiyu Liu},
                % the name of the supervisor
                supervisor={Supervised by: Cyprien Hoelzl, Prof. Eleni Chatzi},
                % the degree programme
                    degree={BSc},
                % the dissertation    title (which cannot be blank)
                     title={Semester Project},
                % the dissertation subtitle (which can    be blank)
                  subtitle={Data-driven identificiation and classification of rail surface defectse},
                % the dissertation     type
                %  type={enterprise},
                % the year of submission
                      year={2019} ]{dissertation}

%
\documentclass[ % the name of the author
                    author={Aiyu Liu},
                % the name of the supervisor
                supervisor={Supervised by: Cyprien Hoelzl, Prof. Eleni Chatzi},
                % the degree programme
                    degree={BSc},
                % the dissertation    title (which cannot be blank)
                     title={Semester Project},
                % the dissertation subtitle (which can    be blank)
                  subtitle={Data-driven identificiation and classification of rail surface defectse},
                % the dissertation     type
                %  type={enterprise},
                % the year of submission
                      year={2019} ]{dissertation}

%
\documentclass[ % the name of the author
                    author={Aiyu Liu},
                % the name of the supervisor
                supervisor={Supervised by: Cyprien Hoelzl, Prof. Eleni Chatzi},
                % the degree programme
                    degree={BSc},
                % the dissertation    title (which cannot be blank)
                     title={Semester Project},
                % the dissertation subtitle (which can    be blank)
                  subtitle={Data-driven identificiation and classification of rail surface defectse},
                % the dissertation     type
                %  type={enterprise},
                % the year of submission
                      year={2019} ]{dissertation}

%\input{config/prelude}



%%% URL line break      
\expandafter\def\expandafter\UrlBreaks\expandafter{\UrlBreaks%  save the current one
  \do\a\do\b\do\c\do\d\do\e\do\f\do\g\do\h\do\i\do\j%
  \do\k\do\l\do\m\do\n\do\o\do\p\do\q\do\r\do\s\do\t%
  \do\u\do\v\do\w\do\x\do\y\do\z\do\A\do\B\do\C\do\D%
  \do\E\do\F\do\G\do\H\do\I\do\J\do\K\do\L\do\M\do\N%
  \do\O\do\P\do\Q\do\R\do\S\do\T\do\U\do\V\do\W\do\X%
  \do\Y\do\Z}
\usepackage{minted}

\usepackage{caption}
\usepackage{amsmath}
\usepackage{todonotes}
\usepackage{amssymb}
%\PassOptionsToPackage{hyphens}{url}
%\usepackage[hidelinks]{hyperref}
\usepackage{diagbox}
\usepackage{algorithm}
\usepackage{graphicx}
\usepackage{subcaption}
\usepackage{makecell}
\usepackage{comment}
\usepackage[noend]{algpseudocode}
%\usepackage[usenames, dvipsnames, svgnames, table]{xcolor}
\usepackage{bm}
\usepackage{array}
\usepackage{pgffor}
\usepackage{changepage}
\usepackage{pbox}
\usepackage{tabularx}
\usepackage{afterpage}
\usepackage{forest} % for making directory trees
\usepackage{emptypage} % get rid of numbering after title page
\usepackage{enumitem}
\usepackage{filecontents}

\usepackage{pgfplotstable} 
\usepackage{booktabs} 
\usepackage{filecontents}

%\usepackage[lite]{mtpro2}
%\newcommand{\todo}[1]{{\color{red} #1}}

\newcommand{\forceindent}{\leavevmode{\parindent=1cm\indent}}
\newcommand{\forceindenthalf}{\leavevmode{\parindent=0.5cm\indent}}

\definecolor{dkgreen}{rgb}{0,0.6,0}
\definecolor{gray}{rgb}{0.5,0.5,0.5}
\definecolor{mauve}{rgb}{0.58,0,0.82}
\lstset{%frame=tb,
%  language=Java,
  aboveskip=3mm,
  belowskip=3mm,
  showstringspaces=false,
  columns=flexible,
  basicstyle={\small\ttfamily},
  numbers=none,
  numberstyle=\tiny\color{gray},
  keywordstyle=\color{blue},
  commentstyle=\color{dkgreen},
  stringstyle=\color{mauve},
  breaklines=true,
  breakatwhitespace=true,
  tabsize=2
}
%backgroundcolor = \color{lightgray}

\newcommand{\specialcell}[2][c]{%
  \begin{tabular}[#1]{@{}c@{}}#2\end{tabular}}



%%% URL line break      
\expandafter\def\expandafter\UrlBreaks\expandafter{\UrlBreaks%  save the current one
  \do\a\do\b\do\c\do\d\do\e\do\f\do\g\do\h\do\i\do\j%
  \do\k\do\l\do\m\do\n\do\o\do\p\do\q\do\r\do\s\do\t%
  \do\u\do\v\do\w\do\x\do\y\do\z\do\A\do\B\do\C\do\D%
  \do\E\do\F\do\G\do\H\do\I\do\J\do\K\do\L\do\M\do\N%
  \do\O\do\P\do\Q\do\R\do\S\do\T\do\U\do\V\do\W\do\X%
  \do\Y\do\Z}
\usepackage{minted}

\usepackage{caption}
\usepackage{amsmath}
\usepackage{todonotes}
\usepackage{amssymb}
%\PassOptionsToPackage{hyphens}{url}
%\usepackage[hidelinks]{hyperref}
\usepackage{diagbox}
\usepackage{algorithm}
\usepackage{graphicx}
\usepackage{subcaption}
\usepackage{makecell}
\usepackage{comment}
\usepackage[noend]{algpseudocode}
%\usepackage[usenames, dvipsnames, svgnames, table]{xcolor}
\usepackage{bm}
\usepackage{array}
\usepackage{pgffor}
\usepackage{changepage}
\usepackage{pbox}
\usepackage{tabularx}
\usepackage{afterpage}
\usepackage{forest} % for making directory trees
\usepackage{emptypage} % get rid of numbering after title page
\usepackage{enumitem}
\usepackage{filecontents}

\usepackage{pgfplotstable} 
\usepackage{booktabs} 
\usepackage{filecontents}

%\usepackage[lite]{mtpro2}
%\newcommand{\todo}[1]{{\color{red} #1}}

\newcommand{\forceindent}{\leavevmode{\parindent=1cm\indent}}
\newcommand{\forceindenthalf}{\leavevmode{\parindent=0.5cm\indent}}

\definecolor{dkgreen}{rgb}{0,0.6,0}
\definecolor{gray}{rgb}{0.5,0.5,0.5}
\definecolor{mauve}{rgb}{0.58,0,0.82}
\lstset{%frame=tb,
%  language=Java,
  aboveskip=3mm,
  belowskip=3mm,
  showstringspaces=false,
  columns=flexible,
  basicstyle={\small\ttfamily},
  numbers=none,
  numberstyle=\tiny\color{gray},
  keywordstyle=\color{blue},
  commentstyle=\color{dkgreen},
  stringstyle=\color{mauve},
  breaklines=true,
  breakatwhitespace=true,
  tabsize=2
}
%backgroundcolor = \color{lightgray}

\newcommand{\specialcell}[2][c]{%
  \begin{tabular}[#1]{@{}c@{}}#2\end{tabular}}



%%% URL line break      
\expandafter\def\expandafter\UrlBreaks\expandafter{\UrlBreaks%  save the current one
  \do\a\do\b\do\c\do\d\do\e\do\f\do\g\do\h\do\i\do\j%
  \do\k\do\l\do\m\do\n\do\o\do\p\do\q\do\r\do\s\do\t%
  \do\u\do\v\do\w\do\x\do\y\do\z\do\A\do\B\do\C\do\D%
  \do\E\do\F\do\G\do\H\do\I\do\J\do\K\do\L\do\M\do\N%
  \do\O\do\P\do\Q\do\R\do\S\do\T\do\U\do\V\do\W\do\X%
  \do\Y\do\Z}
\usepackage{minted}

\usepackage{caption}
\usepackage{amsmath}
\usepackage{todonotes}
\usepackage{amssymb}
%\PassOptionsToPackage{hyphens}{url}
%\usepackage[hidelinks]{hyperref}
\usepackage{diagbox}
\usepackage{algorithm}
\usepackage{graphicx}
\usepackage{subcaption}
\usepackage{makecell}
\usepackage{comment}
\usepackage[noend]{algpseudocode}
%\usepackage[usenames, dvipsnames, svgnames, table]{xcolor}
\usepackage{bm}
\usepackage{array}
\usepackage{pgffor}
\usepackage{changepage}
\usepackage{pbox}
\usepackage{tabularx}
\usepackage{afterpage}
\usepackage{forest} % for making directory trees
\usepackage{emptypage} % get rid of numbering after title page
\usepackage{enumitem}
\usepackage{filecontents}

\usepackage{pgfplotstable} 
\usepackage{booktabs} 
\usepackage{filecontents}

%\usepackage[lite]{mtpro2}
%\newcommand{\todo}[1]{{\color{red} #1}}

\newcommand{\forceindent}{\leavevmode{\parindent=1cm\indent}}
\newcommand{\forceindenthalf}{\leavevmode{\parindent=0.5cm\indent}}

\definecolor{dkgreen}{rgb}{0,0.6,0}
\definecolor{gray}{rgb}{0.5,0.5,0.5}
\definecolor{mauve}{rgb}{0.58,0,0.82}
\lstset{%frame=tb,
%  language=Java,
  aboveskip=3mm,
  belowskip=3mm,
  showstringspaces=false,
  columns=flexible,
  basicstyle={\small\ttfamily},
  numbers=none,
  numberstyle=\tiny\color{gray},
  keywordstyle=\color{blue},
  commentstyle=\color{dkgreen},
  stringstyle=\color{mauve},
  breaklines=true,
  breakatwhitespace=true,
  tabsize=2
}
%backgroundcolor = \color{lightgray}

\newcommand{\specialcell}[2][c]{%
  \begin{tabular}[#1]{@{}c@{}}#2\end{tabular}}

%%%%%%%%%%%%%%%%%%
% BEGIN DOCUMENT %
%%%%%%%%%%%%%%%%%%

%\usepackage[final]{listings}

%\input{csvfiles}

\begin{document}

\maketitle
%\frontmatter
%\makedecl
\newpage
\chapter*{Acknowledgements}
%This semester project would not be possible without the help of my supervisors, Cyprien Hoelzl and professor Eleni Chatzi. I first reached out to professor Chatzi for a semester project opportunity during ETH week 

%Working alongside Cyprien has been a great experience.
%
%I received all the guidance necessary
%Whenever I had issues I could always ask and the reply would come promptly
%provided me with many informative resources
%very good at explaining concepts
%very smart
%and very specialised in this field -- huge understanding
%Can ask any questions, down-to-earth and very helpful. I could not ask for a better supervisor. 
%
%Chatzi is very approachable and kind, good at providing feedback at the intermediate sessions.
%


%I reached out to Ian from literally the other side of the world, when I was still an exchange student at Tsinghua University in Beijing. When no professors seemed to be interested in supervising me for my final capstone project, he was happy to do so. He even offered to do a Skype call with me to discuss the available options for the project. I am truly grateful for his generosity in offering to supervise me.
%Ever since coming to Bristol University for exchange, he arranged weekly meetings to discuss project updates, answer any troubling questions, suggest important litterature and provide general clarifications. In general, great academic guidance was received and time were always made available if additional questions arose. These weekly meetings were very helpful and kept me accountable throughout the entire process.
%I appreciate his welcoming character, his attentiveness, rigorous scholarship and his ability to clearly explain answers to my questions. He truly is an expert in the field of visualisation and pattern recogni- tion; a master of his craft. I am truly humbled by his expertise and I have great respect for Professor Nabney.


\newpage



%\include{abstract}
%\mainmatter
%\setcounter{tocdepth}{3}
%\setcounter{secnumdepth}{3} dunno what this does.
{
  \hypersetup{linkcolor=black}
  \tableofcontents
  % Two times compilation
  %\listoffigures
  %\listoftables
   
}
\newpage
%%%%%%%%%%%%%%%%%%%%%%%%%%%%%%%%%%%%%%%%%%%%%%%%%%%%%%%%%%%%%%%%%%%%%%%%%%%%%%%%%%%%%%%%%%%%%%%%
%%% Intro
%%%%%%%%%%%%%%%%%%%%%%%%%%%%%%%%%%%%%%%%%%%%%%%%%%%%%%%%%%%%%%%%%%%%%%%%%%%%%%%%%%%%%%%%%%%%%%%%
\chapter{Introduction}
%\section{Problem description and motivation} \todo{maybe remove this section}
Railway companies need to continuously and sufficiently maintain the train tracks and optimally detect defects in order to have a more punctual and more effective train system. However, the current system is expensive, time consuming and ineffective. That is, maintenance agents need to walk along tracks and check them for defects. For visualisation purposes, there is roughly 5200 km of rails in Switzerland which needs to be inspected by 40 experienced inspectors.


In order to cope with this issue, Swiss Federal Railways (SBB) has specifically built two new special diagnostic vehicles (SDV) designed for defect identification among other purposes. For this, accelerometers have been installed at the front and back of the SDV to collect the signal responses from the wheel and the train track (see appendix \ref{figs:veh}).

A defect in train tracks can be seen as a discontinuity. As a train passes over this discontinuity, it will result in a perturbation that can be detected by sensors. It is our main assumption that each type of defect will have a specific signature that will allow its identification and classification. This is similar to the idea presented in \cite{Introduc31:online} about human activity recognition.

\section{Objective}
As the title implies, the objective of this project is to identify and classify rail surface defects. To do this, we aim to build an effective pipeline that takes information about defects as input and outputs a classification confidence for these defects. By succesfully identifying and classifying the defects, we take one step further towards reducing delays and making trains more punctual and reliable. In this development, the first step consists of identification and classification, while the second step ultimately consists of future defect prediction. 

\section{Defects}
Evidently, a defect can be seen as a deviation from the standard train track. For the exact defect type, SBB has a database for the individual defect definitions. See appendix \ref{app:report} for an example as to how this is done.
%Gleis	149.849086	(400, A-184063)	
%chienenzwischenlage	4.989620	(400, A-146358)	
%Bankett	1169.510537	(400, A-103213)

%Schwelle	0.000000	(400, A-103206)	
%Schwelle	0.000000	(400, A-118152)	
%Vignolschiene	0.000000	(400, A-231400)	
%Fahrbahn	0.000000	(400, A-233599)	

Generally, a defect is separated into two overarching types: range- and point-defects. I.e. a defect that is detected at a single point versus a defect that is detected at varying lengths. A point defect is perceived as a sharp signal response, whereas a range-defect is perceived over a greater time period. 
\begin{figure}[H]
	\centering
	%\includegraphics[width = 0.49\textwidth]{imgs/defs/p(400,A-103206)0}
	%\caption{Defect ID: A-$103206$; Report set ID: $400$}
	\includegraphics[width = 0.2\textwidth]{imgs/defs/p(400,A-118152)0}
	\includegraphics[width = 0.38\textwidth]{imgs/defs/p(400,A-231400)0}
	\includegraphics[width = 0.38\textwidth]{imgs/defs/p(400,A-233599)0}
	\caption{Left, middle, right: Schwelle (A-$118152$), Fahrbahn (A-$233599$), Vignolschiene (A-$231400$). These have all been reported as defects (with subcategories) with a length equals to zero, and thus have been classified as point defects using our terminology. Parenthesis signifies defect ID and all pictures comes from SBB defect report set ID: $400$}
\end{figure}
%\vspace*{-1cm}
\begin{figure}[H]
	\centering
	%\includegraphics[width = 0.49\textwidth]{imgs/defs/p(400,A-103206)0}
	%\caption{Defect ID: A-$103206$; Report set ID: $400$}
	\includegraphics[width = 0.33\textwidth]{imgs/defs/r(400,A-184063)0}
	\includegraphics[width = 0.33\textwidth]{imgs/defs/r(400,A-184063)1}
	
	\includegraphics[width = 0.33\textwidth]{imgs/defs/r(400,A-146358)0}
	\includegraphics[width = 0.33\textwidth]{imgs/defs/r(400,A-146358)1}
\end{figure}
\begin{figure}[H]
	\centering
	\includegraphics[width = 0.33\textwidth]{imgs/defs/r(400,A-103213)0}
	\caption{Top, middle, bottom: Gleis-$149.8 m$ (A-$184063$), Schienenzwischenlage-$5.0 m$ (A-$146358$), Bankett-$1169.5 m$ (A-$231400$). These have all been reported as defects (with subcategories) with a length strictly greater than zero, and thus have been classified as range defects using our terminology. Parenthesis signifies defect ID and all pictures comes from SBB defect report set ID: $400$. Most of these range defects have two pictures likely to give more detail}
\end{figure}
\raggedbottom % to get of unnecessary vertical spacing
For this project, we have solely focused on the point defects for analysis, as this simplifies the problem statement. We thus disregard range-defects such that we do not have to deal with the extra, associated factors. 

\section{Switches and insulation joints}

\begin{figure}[H]
	\centering
	\includegraphics[width = 0.49\textwidth]{imgs/switch0}
	\label{fig:veh0}
	\includegraphics[width = 0.49\textwidth]{imgs/switch1}
	\label{fig:veh1}
	\caption{Right: \url{thesprucecrafts.com/model-train-switches-2382606}, 
	Left: 	\url{https://www.indiamart.com/proddetail/railroad-switches-3879200755.html}}
	\label{fig:veh2}
\end{figure}
\begin{figure}[H]
	\centering
	\includegraphics[width = 0.49\textwidth]{imgs/ins_joint0}
	\label{fig:veh0}
	\includegraphics[width = 0.49\textwidth]{imgs/ins_joint1}
	\label{fig:veh1}
	\caption{Right: \url{http://www.railroad-fasteners.com/news/Insulated-Rail-Joint.html}, Left: \url{http://www.railroadpart.com/rail-joints/insulated-rail-joints.html}}
	\label{fig:veh2}
\end{figure}

insert text \todo{}
\section{Data}
The data has been collected and provided by SBB. Using their SDV, SBB has made trips back and forth to different cities in Switzerland in order to collect various data including but not limited to accelerometer data. After getting the data from SBB, it then goes through a processing pipeline (designed by Cyprien), after which the data can be manipulated with \verb|python| dataframes (from \verb|pd.DataFrame|). The accelerometer captures the accelerations at the XYZ-axes (along with the timestamps at each recording), of which we are only concerned with the Y-axis for the vertical pertubations for the accelerometer at the axle. See appendix \ref{figs:veh} for visualisations of the accelerometer placements on the SDV. 

These are the measurement rides that we are dealing with
%new_filenames = ['__new_gDFZ_2019-05-27T08_55_55_819Z_DFZ01', # s - 819Z_DFZ01
%                 '__new_gDFZ_2019-05-27T10_03_59_077Z_DFZ01', # t - 077Z_DFZ01
%                 '__new_gDFZ_2019-05-27T13_05_53_330Z_DFZ01', # u - 330Z_DFZ01
%                 '__new_gDFZ_2019-05-27T14_10_51_425Z_DFZ01'] # v - 425Z_DFZ01
\begin{table}[H]
	\centering
	\begin{tabular}{|c|c|c|c|} \hline
		\textbf{From} & \textbf{To} & \textbf{Date} & \textbf{Campaign ID}\\ \hline \hline 
		- & - &  \verb|2019-05-27T08_55_55| & \verb|819Z DFZ01| \\ \hline 
		- & - &  \verb|2019-05-27T10_03_59| & \verb|077Z DFZ01| \\ \hline 
		- & - &  \verb|2019-05-27T13_05_53| & \verb|330Z DFZ01| \\ \hline 
		- & - &  \verb|2019-05-27T14_10_51| & \verb|425Z DFZ01| \\ \hline 

	\end{tabular}
	\caption{Measurement rides}
	\label{datasets}
\end{table}

Furhtermore, the locations of the defects have to be retrieved from SBB's database. which were retrieved by Cyprien. 

In this report, we will use the word 'entity' as an umbrella term for the different track entities: switch, insulation joint and defect.


\section{Code}
\label{int:sec:code}
The code is written purely in python. To create neural network architectures, we are using: \verb|keras| along with \verb|tensorflow|. \verb|keras| is essentially a high-level neural networks library which runs on top of \verb|tensorflow|. It has a consistent, simple API and provides clear and actionable feedback upon user error. Models are easily made by connecting configurable building blocks together, with few restrictions \cite{TensorFl31:online}. The models were trained in \verb|Google Colab|, which is a web application provided by Google that enables users to run python code in the web browser with access to GPUs\footnote{\url{https://colab.research.google.com/notebooks/intro.ipynb}}. It is very similar to \verb|Anaconda's| \verb|Jupyter Notebooks|, except that \verb|Colab| runs in the browser, is collaborative and provides free usage of GPUs (meaning model training goes faster).\\


\noindent All the code can be found on github: \\
\url{https://github.com/Aiyualive/SemesterProject2.0}.\\

\noindent The specific model execution workflow can be found \verb|Colab|:\\
\url{https://colab.research.google.com/drive/12VBz_KrJxeyR_pjpkC87fewv5aMSEI5_}

%mainly uses pandas and stuff


%%%%%%%%%%%%%%%%%%%%%%%%%%%%%%%%%%%%%%%%%%%%%%%%%%%%%%%%%%%%%%%%%%%%%%%%%%%%%%%%%%%%%%%%%%%%%%%%
%%% Design and implementation
%%%%%%%%%%%%%%%%%%%%%%%%%%%%%%%%%%%%%%%%%%%%%%%%%%%%%%%%%%%%%%%%%%%%%%%%%%%%%%%%%%%%%%%%%%%%%%%%
\chapter{Design and Implementation}

For the process of defect classification we designed the pipeline in \ref{fig:pipeline}. In the next sections, I would like to give an overview of how each step was implemented.
\begin{figure}[H]
	\centering
	\includegraphics[width = \textwidth]{imgs/pipeline}
	\caption{Primary pipeline}
	\label{fig:pipeline}
\end{figure}


%--> reevaluation| Realised that the results were not very good. So we need to visualise the data and look for separability to make a better model


\section{Shift of GPS timestamps}
The SDV has its GPS sensor installed at a specified location on the vehicle body. However, what we need to achieve is the position (covered distance) at each accelerometer at either sides of the GPS. Since the GPS sensor is sampled at a lower frequency compared to the accelerometers (every $25$ cm vs $24$ kHz respectively), we first need to get the corresponding positions for each accelerometer sample. This is done by interpolation using the timestamps of the accelerometers and GPS. 

Depending on the direction of the vehicle we then subtract/add the offset between the accelerometers and the GPS sensor with regard to the position of these sensors on the vehicle body. \todo{insert drawing of how it is calculated?}

\section{Peak windows}
Retrieving the signal response around the defect location forms a crucial aspect in the overarching pipeline. The goal of this step is to, around each defect, create a "window" containing accelerometer accelerations of a specified time length -- wherein Within the highest acceleration recording around is found in the center. As a result, all of these windows would be uniform in the sense that they are all centered according to the highest recording of a defect. It is then assumed that each window forms the signature of each track entity. 

Since we are assuming that each track entity is identified by a well-formed peak, we first need to find this peak within a reasonable offset from the defect location, after which we center around that within another reasonable offset. 

In the code, this is done by defining two parameters: \verb|find_peak_offset = 1| and \verb|window_offset = 0.5|. I.e. given a defect timestamp, we search for the highest acceleration recording that has occured $1$ second after and $1$ second before the defect timestamp. Once the peak has been found, we then center it in a $1$ second window ($0.5$ sec on each side). \todo{insert a drawing of how this works?}

\section{Entity library}
The peak windows arguably forms the central feature of the defect library. However, based on domain knowledge, other features like speed also needs to be considered for our neural network. So apart from the peak windows, we have also extracted other of relevant features that might be useful for classifcation:
%timestamps	accelerations	window_length(s)	severity	vehicle_speed(m/s)	axle	campagin_ID	driving_direction	ID	class_labe
%timestamps	accelerations	window_length(s)	severity	vehicle_speed(m/s)	axle	campagin_ID	driving_direction	defect_type	defect_length(m)	line, defect_ID	class_label
\begin{itemize}
	\item \textbf{Timestamps:} timestamps for the sampled acceleration
	\item \textbf{Acceleration:} sampled acceleration at axle box.
	\item \textbf{Vehicle speed (m/s):} vehicle speed at the closest timestamp
	\item \textbf{Severity:} Severity of entity; defects have an integer label from $1-4$ whereas both insulation joints and switches have the label $5$
\end{itemize}
Additional information about each entity has been retrieved as well, such as: driving direction and corresponding entity IDs. For each entity, we crucially set a true, class label such that we are able to do supervised learning.

Given a specific measurement ride object (handled by Cyprien), we either retrieve each feature directly from the corresponding \verb|dataframe| or with the use of designated helper functions for those requring extra processing. Currently, we have have a 2-level nested \verb|for loop|, looping for each axle outerly, and looping for each entity entry innerly.

The implementation of this could have made more elegant by operating directly on the dataframe, which might also increase speed of the implementation as the \verb|pandas| library has optimised their dataframe operations. However, speed and efficency was not a major concern in this project.

show a few entity signals and their features, refer to the defects presented in introduction,  appendix for more signals? \todo{}

\section{Classification}
We have created a primary \verb|NN| class (short for neural network) along with a \verb|ModelMaker| class. The former does everything from pre-processing the data to evaluating the used model. The latter, as the name suggests, is utlised for creating and using different models, which is useful as we can keep track of how the models have been modified and improved. 

\subsection{NN class}
To make a classification, we first need to select the relevant features. Then we simply feed the features into an \verb|NN| object, where the API of the \verb|NN| class can be called for classification. The usage of the \verb|NN| class is demonstrated below in \ref{tab:nnclass}.


\begin{table}[H]
	\centering
	\small
	\begin{tabular}{l p{0.5\linewidth}}\noalign{\global\arrayrulewidth=0.3mm} 
	\hline 
	\multicolumn{2}{c}{\textbf{API of NN class}}
		\\ \hline  
		\noalign{\global\arrayrulewidth=0.05mm}
		\verb|__init__()|                  & initialises a \verb|NN| object\\ \hline
		\verb|prepare_data()|            & pre-process data, this includes standardisation of data\\ \hline
		\verb|make_model()|              & uses \verb|ModelMaker| class to select a model \\ \hline
		
		\verb|fit()|			         & trains the model\\ \hline
		\verb|classify()|			     & classifies on an eventual test set\\ 
		\noalign{\global\arrayrulewidth=0.3mm} \hline 
		
		
		\vspace*{0.25cm}\\
		\hline
		\multicolumn{2}{c}{ \textbf{Other utility API functions} }	
		\\ \hline 
		\noalign{\global\arrayrulewidth=0.05mm}
		\verb|measure_performance()| & currently only on validation data \\ \hline
		\verb|plot_metrics()|          \\  \hline
		\verb|plot_confusion_matrix()| \\ \hline
		\verb|load_weights()|             \\ \hline
		\verb|load_model_()|             \\ \hline
		\verb|save_history()|           \\ \hline
		\verb|save_model()|              \\ \hline
		\verb|save_classification_to_csv()|  \\ \hline
		\noalign{\global\arrayrulewidth=0.3mm}
		\hline \hline
		\verb|run_experiment()| & evaluates the given model for a \# of repetitions\\ 
		\noalign{\global\arrayrulewidth=0.3mm}
		\hline
	\end{tabular}
	\caption{To train a model, the first API functions needs to be called sequentially. Other utility functions are rather self-explanatory.}\label{tab:nnclass}
\end{table}
\raggedbottom % To get rid of weird spacing
%\todo{insert into appendix?}

\subsection{ModelMaker class}
As mentioned in the introduction \ref{int:sec:code}, this is where we make use of \verb|keras|.

See example of this in next chapter. \todo{todo, explan each layer?}

\section{Visualisation}
Finally, after evaluating the results (results can be seen in the next section) from the neural network, we have not achieved any significant results. Arguably, the visualisations of class separability should have been handled first. However, the previous steps took the majority of the time.\todo{todo}

%%%%%%%%%%%%%%%%%%%%%%%%%%%%%%%%%%%%%%%%%%%%%%%%%%%%%%%%%%%%%%%%%%%%%%%%%%%%%%%%%%%%%%%%%%%%%%%%
%%% Evaluation
%%%%%%%%%%%%%%%%%%%%%%%%%%%%%%%%%%%%%%%%%%%%%%%%%%%%%%%%%%%%%%%%%%%%%%%%%%%%%%%%%%%%%%%%%%%%%%%%
\chapter{Evaluation}
Here we will present the results and discuss the findings herein.


\section{Models}
\begin{table}[H]
	\centering
	\begin{tabular}{|c|c|c| c|} \hline
		\textbf{Layer} & \textbf{Output Shape } & \textbf{Number of params} \\ \hline \hline 
		- & - &  -  \\ \hline 
	\end{tabular}
	\caption{Model 1}
	\label{datasets}
\end{table}

Draw models \url{http://alexlenail.me/NN-SVG/AlexNet.html} \todo{do one model at a time}

\section{Model evaluations}

\begin{table}[H]
	\centering
	\begin{tabular}{|c|c|c|} \hline
		\textbf{Defect Type} & $2$ & $ \%$\\ \hline
	\end{tabular}
	\caption{Entity distribution, class distributions}
\end{table}

In this section we evaluate our model with regard to a variety of metrics: loss (\textbf{L}), accuracy (\textbf{ACC}), true positives (\textbf{TP}), false positives (\textbf{FP}), true negatives (\textbf{TN}), false negatives (\textbf{FN}), precision (\textbf{P}), recall (\textbf{R}), area under the curve (\textbf{AUC}). Insert explanation of each metric\todo{?}
%&{L}&{ACC}&{TP}&{TN}&{FP}&{FN}&{P}&{R}&{AUC}
\begin{table}[H]
		\centering 
		\begin{tabular}{|l||*{5}{c|}}\hline
			\backslashbox{Metric}{Model}
			&\makebox[6.3em]{Model 1}&\makebox[6.3em]{}&\makebox[6.3em]{}&\makebox[6.3em]{} 
			\\\hline\hline
			L   & $-$ & $-$ & $-$ & $-$ \\ \hline
			ACC & $-$ & $-$ & $-$ & $-$ \\ \hline
			TP  & $-$ & $-$ & $-$ & $-$ \\ \hline
			TN  & $-$ & $-$ & $-$ & $-$ \\ \hline
			FP  & $-$ & $-$ & $-$ & $-$ \\ \hline
			FN  & $-$ & $-$ & $-$ & $-$ \\ \hline
			P   & $-$ & $-$ & $-$ & $-$ \\ \hline
			R   & $-$ & $-$ & $-$ & $-$ \\ \hline
			AUC & $-$ & $-$ & $-$ & $-$ \\ \hline
			\Xhline{3\arrayrulewidth} 
			Relative diff? & $-$ & $-$ & $-$ & $-$ \\ \hline 
		\end{tabular}
		\captionof{table}{Average metrics times and their standard deviations in parenthesis - rel diff?}
		\label{tab:resBenchmark}
\end{table} 

average epoch plot?


\section{Visualisation of class clustering}
insert the pca plots

\section{Discussion}
Data amount, circumvent: could self-engineer data. 

should have done visualisation first, if we have clear cluster separation, applying a neural network would be a bit exaggerated. And in that case, we could opt for a simple multi class support vector machine from the \verb|sklearn| library. However, using \verb|tensorflow| was the plan from the get-go as it is more industrially-applicable, so we disregarded simpler methods.

ensure that data is uniform. That is, some of the data has calibration and some hasnt. 




%I tried to increase the outliers, but this was a hugely naive approach

%%%%%%%%%%%%%%%%%%%%%%%%%%%%%%%%%%%%%%%%%%%%%%%%%%%%%%%%%%%%%%%%%%%%%%%%%%%%%%%%%%%%%%%%%%%%%%%%
%%% Conclusion and future work
%%%%%%%%%%%%%%%%%%%%%%%%%%%%%%%%%%%%%%%%%%%%%%%%%%%%%%%%%%%%%%%%%%%%%%%%%%%%%%%%%%%%%%%%%%%%%%%%
\chapter{Conclusion and future work}

\section{Conclusion}
Results were quite mediocre, but has a lot of room for improvement. I am sure that given more time I would be able to explore and evaluate the results further.

how good is the foundation to move onwards with further research
\section{Future work}

\begin{itemize}
	\item Might be interesting to also consider the XZ-axes.
	\item range defects
	\item tune the peak finding parameters
	\item track entity dependent/specific window offsets
	\item we must not set the findpeakoffset too high
	\item Questions, what if you want to use multiple features with 1 CNN
	\item I have participated in ETH Hatchery (ref), where our team build a prototype with a model train. We let the model train drive on the track with self-engineered defects.

\end{itemize}

\newpage
\section{TODO}
\begin{itemize}
	\item get percentage of each class in the validation set
	\item create an average of the model ie run model for more times
	\item add speed as a feature
	\item what does each filter do? what is kernel size?
	\item very fast speed, overlap between switch and ins, old vs new rail, ax1 arrow 2 arrow 3 arrow 4
	\item 3D plots?
	\item change the defect library to use pandas instead?
	\item visualise what the network is doing using Harry's code
	\item use speed as a feature also
	\item be consistent with function naming and variable names: function names with underscore and variable names with camelcase
	\item Which type of defects are we actually working with, we can see that it does good at the switches and ins joints, but no chance with the defects
	\item we should instead call it an entity library -- make up your mind
	\item try only with defects and no ins and switches
	\item separate all the axle channels and train on them
	\item try to use low pass filter
	\item get better results
	\item we dont need non-defects for defect classification, we could just input something else and make sure that it is not a defect
	\item a specifc defect type vs ins joint vs switch
	\item we did not consider severity
	\item which track entities are we actually analysing
	\item save the class distribution as a text file
	\item unique bincount of each defect severity
	\item insert signal plots, uniform axes. dont do peakfinding
	\item better epoch plots
\end{itemize}

\bibliography{master}
%\todo{New paper with train}

\newpage
\cleardoublepage
\appendix
%%%%%%%%%%%%%%%%%%%%%%%%%%%%%%%%%%%%%%%%%%%%%%%%%%%%%%%%%%%%%%%%%%%%%%%%%%%%%%%%%%%%%%%%%%%%%%%%
%%% Appendix introduction
%%%%%%%%%%%%%%%%%%%%%%%%%%%%%%%%%%%%%%%%%%%%%%%%%%%%%%%%%%%%%%%%%%%%%%%%%%%%%%%%%%%%%%%%%%%%%%%
\chapter{Introduction}
\section{List of defect categories}
herstuck 

schiene

etc

maybe just retrieve from the SBB reports


\section{Vehicle and accelerometer placements}
\label{figs:veh}
\begin{figure}[H]
	\centering
	\includegraphics[width = 0.58\textwidth]{imgs/veh0}
	\label{fig:veh0}
	
	\includegraphics[width = 0.58\textwidth]{imgs/veh1}
	\label{fig:veh1}
	
	\includegraphics[width = 0.58\textwidth]{imgs/veh2}
	\caption{These figures have been provided by Cyprien's folder of SBB documents. They show the accelerometer placements. For this project we have only considered axle number 4 in the third figure.}
	\label{fig:veh2}
\end{figure}

\section{SBB defect report example}
\label{app:report}
\begin{figure}[H]
	\centering
	\includegraphics[width = 0.9\textwidth]{imgs/defs/rep0}
	\caption{}
	\includegraphics[width = 0.9\textwidth]{imgs/defs/rep1}
	\caption{A typical report for an arbitrary defect usually contains one description page followed by its picture(s)}
	\label{fig:veh2}
\end{figure}


%%%%%%%%%%%%%%%%%%%%%%%%%%%%%%%%%%%%%%%%%%%%%%%%%%%%%%%%%%%%%%%%%%%%%%%%%%%%%%%%%%%%%%%%%%%%%%%%
%%% Appendix implementation
%%%%%%%%%%%%%%%%%%%%%%%%%%%%%%%%%%%%%%%%%%%%%%%%%%%%%%%%%%%%%%%%%%%%%%%%%%%%%%%%%%%%%%%%%%%%%%%
\chapter{Evaluation}
\begin{table}[H]
	\centering
\begin{tabular}{|l|ccccccccc|}
	\hline
	\backslashbox{run \#}{Metrics}
&{L}&{ACC}&{TP}&{TN}&{FP}&{FN}&{P}&{R}&{AUC} \\\hline\hline
	1 & $-$ & $-$ & $-$ & $-$ & $-$ & $-$ & $-$ & $-$ & $-$ \\ \hline
\end{tabular}
	\caption{Experiment result of 10 runs}
\end{table}

confusion matrix, epoch plots

%\begin{table}[H]
%	\centering
%	\begin{tabular}{|c|c|} \hline
%		\textbf{Latent Dimension} & $2$ \\ \hline
%		\textbf{Shape} & 8 $\times$ $8$ \\ \hline
%		\textbf{Latent Points} &  $64$\\ \hline
%		\textbf{Dist. Mixtures} &  Bernoulli\\ \hline \hline
%		\textbf{Regularisation $\alpha$ } & $0.001$ \\ \hline
%	\end{tabular}
%	\hspace*{0.5cm}
%	\begin{tabular}{|l|c|} \hline
%		\textbf{RBF Dimension} & $2$ \\ \hline
%		\textbf{RBF grid}  & $4\times4$ \\ \hline 
%		\textbf{RBF centres} & $16$ \\ \hline
%		\textbf{Basis function} & Gaussian \\ \hline
%	    \textbf{Widths $\bm{\sigma}$} & $1.0$\\ \hline
%	\end{tabular}
%	\caption{The parameters for the latent model and the RBF neural network. Distribution mixtures and basis functions cannot be altered as our implementation is specific for the Bernoulli GTM version}
%	\label{eval:model}
%\end{table}


%\label{code}
%\inputminted[linenos, fontsize = \scriptsize]{python}{../src/utils/defect_utils.py}

\end{document}
