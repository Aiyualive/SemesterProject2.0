%\documentclass[11pt]{report}
%\usepackage[a4paper]{geometry}

\documentclass[ % the name of the author
                    author={Aiyu Liu},
                % the name of the supervisor
                supervisor={Supervised by: Cyprien Hoelzl, Prof. Eleni Chatzi},
                % the degree programme
                    degree={BSc},
                % the dissertation    title (which cannot be blank)
                     title={Semester Project},
                % the dissertation subtitle (which can    be blank)
                  subtitle={Data-driven identificiation and classification of rail surface defectse},
                % the dissertation     type
                %  type={enterprise},
                % the year of submission
                      year={2019} ]{dissertation}

%
\documentclass[ % the name of the author
                    author={Aiyu Liu},
                % the name of the supervisor
                supervisor={Supervised by: Cyprien Hoelzl, Prof. Eleni Chatzi},
                % the degree programme
                    degree={BSc},
                % the dissertation    title (which cannot be blank)
                     title={Semester Project},
                % the dissertation subtitle (which can    be blank)
                  subtitle={Data-driven identificiation and classification of rail surface defectse},
                % the dissertation     type
                %  type={enterprise},
                % the year of submission
                      year={2019} ]{dissertation}

%
\documentclass[ % the name of the author
                    author={Aiyu Liu},
                % the name of the supervisor
                supervisor={Supervised by: Cyprien Hoelzl, Prof. Eleni Chatzi},
                % the degree programme
                    degree={BSc},
                % the dissertation    title (which cannot be blank)
                     title={Semester Project},
                % the dissertation subtitle (which can    be blank)
                  subtitle={Data-driven identificiation and classification of rail surface defectse},
                % the dissertation     type
                %  type={enterprise},
                % the year of submission
                      year={2019} ]{dissertation}

%\input{config/prelude}



%%% URL line break      
\expandafter\def\expandafter\UrlBreaks\expandafter{\UrlBreaks%  save the current one
  \do\a\do\b\do\c\do\d\do\e\do\f\do\g\do\h\do\i\do\j%
  \do\k\do\l\do\m\do\n\do\o\do\p\do\q\do\r\do\s\do\t%
  \do\u\do\v\do\w\do\x\do\y\do\z\do\A\do\B\do\C\do\D%
  \do\E\do\F\do\G\do\H\do\I\do\J\do\K\do\L\do\M\do\N%
  \do\O\do\P\do\Q\do\R\do\S\do\T\do\U\do\V\do\W\do\X%
  \do\Y\do\Z}
\usepackage{minted}

\usepackage{caption}
\usepackage{amsmath}
\usepackage{todonotes}
\usepackage{amssymb}
%\PassOptionsToPackage{hyphens}{url}
%\usepackage[hidelinks]{hyperref}
\usepackage{diagbox}
\usepackage{algorithm}
\usepackage{graphicx}
\usepackage{subcaption}
\usepackage{makecell}
\usepackage{comment}
\usepackage[noend]{algpseudocode}
%\usepackage[usenames, dvipsnames, svgnames, table]{xcolor}
\usepackage{bm}
\usepackage{array}
\usepackage{pgffor}
\usepackage{changepage}
\usepackage{pbox}
\usepackage{tabularx}
\usepackage{afterpage}
\usepackage{forest} % for making directory trees
\usepackage{emptypage} % get rid of numbering after title page
\usepackage{enumitem}
\usepackage{filecontents}

\usepackage{pgfplotstable} 
\usepackage{booktabs} 
\usepackage{filecontents}

%\usepackage[lite]{mtpro2}
%\newcommand{\todo}[1]{{\color{red} #1}}

\newcommand{\forceindent}{\leavevmode{\parindent=1cm\indent}}
\newcommand{\forceindenthalf}{\leavevmode{\parindent=0.5cm\indent}}

\definecolor{dkgreen}{rgb}{0,0.6,0}
\definecolor{gray}{rgb}{0.5,0.5,0.5}
\definecolor{mauve}{rgb}{0.58,0,0.82}
\lstset{%frame=tb,
%  language=Java,
  aboveskip=3mm,
  belowskip=3mm,
  showstringspaces=false,
  columns=flexible,
  basicstyle={\small\ttfamily},
  numbers=none,
  numberstyle=\tiny\color{gray},
  keywordstyle=\color{blue},
  commentstyle=\color{dkgreen},
  stringstyle=\color{mauve},
  breaklines=true,
  breakatwhitespace=true,
  tabsize=2
}
%backgroundcolor = \color{lightgray}

\newcommand{\specialcell}[2][c]{%
  \begin{tabular}[#1]{@{}c@{}}#2\end{tabular}}



%%% URL line break      
\expandafter\def\expandafter\UrlBreaks\expandafter{\UrlBreaks%  save the current one
  \do\a\do\b\do\c\do\d\do\e\do\f\do\g\do\h\do\i\do\j%
  \do\k\do\l\do\m\do\n\do\o\do\p\do\q\do\r\do\s\do\t%
  \do\u\do\v\do\w\do\x\do\y\do\z\do\A\do\B\do\C\do\D%
  \do\E\do\F\do\G\do\H\do\I\do\J\do\K\do\L\do\M\do\N%
  \do\O\do\P\do\Q\do\R\do\S\do\T\do\U\do\V\do\W\do\X%
  \do\Y\do\Z}
\usepackage{minted}

\usepackage{caption}
\usepackage{amsmath}
\usepackage{todonotes}
\usepackage{amssymb}
%\PassOptionsToPackage{hyphens}{url}
%\usepackage[hidelinks]{hyperref}
\usepackage{diagbox}
\usepackage{algorithm}
\usepackage{graphicx}
\usepackage{subcaption}
\usepackage{makecell}
\usepackage{comment}
\usepackage[noend]{algpseudocode}
%\usepackage[usenames, dvipsnames, svgnames, table]{xcolor}
\usepackage{bm}
\usepackage{array}
\usepackage{pgffor}
\usepackage{changepage}
\usepackage{pbox}
\usepackage{tabularx}
\usepackage{afterpage}
\usepackage{forest} % for making directory trees
\usepackage{emptypage} % get rid of numbering after title page
\usepackage{enumitem}
\usepackage{filecontents}

\usepackage{pgfplotstable} 
\usepackage{booktabs} 
\usepackage{filecontents}

%\usepackage[lite]{mtpro2}
%\newcommand{\todo}[1]{{\color{red} #1}}

\newcommand{\forceindent}{\leavevmode{\parindent=1cm\indent}}
\newcommand{\forceindenthalf}{\leavevmode{\parindent=0.5cm\indent}}

\definecolor{dkgreen}{rgb}{0,0.6,0}
\definecolor{gray}{rgb}{0.5,0.5,0.5}
\definecolor{mauve}{rgb}{0.58,0,0.82}
\lstset{%frame=tb,
%  language=Java,
  aboveskip=3mm,
  belowskip=3mm,
  showstringspaces=false,
  columns=flexible,
  basicstyle={\small\ttfamily},
  numbers=none,
  numberstyle=\tiny\color{gray},
  keywordstyle=\color{blue},
  commentstyle=\color{dkgreen},
  stringstyle=\color{mauve},
  breaklines=true,
  breakatwhitespace=true,
  tabsize=2
}
%backgroundcolor = \color{lightgray}

\newcommand{\specialcell}[2][c]{%
  \begin{tabular}[#1]{@{}c@{}}#2\end{tabular}}



%%% URL line break      
\expandafter\def\expandafter\UrlBreaks\expandafter{\UrlBreaks%  save the current one
  \do\a\do\b\do\c\do\d\do\e\do\f\do\g\do\h\do\i\do\j%
  \do\k\do\l\do\m\do\n\do\o\do\p\do\q\do\r\do\s\do\t%
  \do\u\do\v\do\w\do\x\do\y\do\z\do\A\do\B\do\C\do\D%
  \do\E\do\F\do\G\do\H\do\I\do\J\do\K\do\L\do\M\do\N%
  \do\O\do\P\do\Q\do\R\do\S\do\T\do\U\do\V\do\W\do\X%
  \do\Y\do\Z}
\usepackage{minted}

\usepackage{caption}
\usepackage{amsmath}
\usepackage{todonotes}
\usepackage{amssymb}
%\PassOptionsToPackage{hyphens}{url}
%\usepackage[hidelinks]{hyperref}
\usepackage{diagbox}
\usepackage{algorithm}
\usepackage{graphicx}
\usepackage{subcaption}
\usepackage{makecell}
\usepackage{comment}
\usepackage[noend]{algpseudocode}
%\usepackage[usenames, dvipsnames, svgnames, table]{xcolor}
\usepackage{bm}
\usepackage{array}
\usepackage{pgffor}
\usepackage{changepage}
\usepackage{pbox}
\usepackage{tabularx}
\usepackage{afterpage}
\usepackage{forest} % for making directory trees
\usepackage{emptypage} % get rid of numbering after title page
\usepackage{enumitem}
\usepackage{filecontents}

\usepackage{pgfplotstable} 
\usepackage{booktabs} 
\usepackage{filecontents}

%\usepackage[lite]{mtpro2}
%\newcommand{\todo}[1]{{\color{red} #1}}

\newcommand{\forceindent}{\leavevmode{\parindent=1cm\indent}}
\newcommand{\forceindenthalf}{\leavevmode{\parindent=0.5cm\indent}}

\definecolor{dkgreen}{rgb}{0,0.6,0}
\definecolor{gray}{rgb}{0.5,0.5,0.5}
\definecolor{mauve}{rgb}{0.58,0,0.82}
\lstset{%frame=tb,
%  language=Java,
  aboveskip=3mm,
  belowskip=3mm,
  showstringspaces=false,
  columns=flexible,
  basicstyle={\small\ttfamily},
  numbers=none,
  numberstyle=\tiny\color{gray},
  keywordstyle=\color{blue},
  commentstyle=\color{dkgreen},
  stringstyle=\color{mauve},
  breaklines=true,
  breakatwhitespace=true,
  tabsize=2
}
%backgroundcolor = \color{lightgray}

\newcommand{\specialcell}[2][c]{%
  \begin{tabular}[#1]{@{}c@{}}#2\end{tabular}}

%%%%%%%%%%%%%%%%%%
% BEGIN DOCUMENT %
%%%%%%%%%%%%%%%%%%

%\usepackage[final]{listings}

%\input{csvfiles}

\begin{document}

\maketitle
%\frontmatter
%\makedecl
\newpage
\chapter*{Acknowledgements}
%This semester project would not be possible without the help of my supervisors, Cyprien Hoelzl and professor Eleni Chatzi. I first reached out to professor Chatzi for a semester project opportunity during ETH week 

%Working alongside Cyprien has been a great experience.
%
%I received all the guidance necessary
%Whenever I had issues I could always ask and the reply would come promptly
%provided me with many informative resources
%very good at explaining concepts
%very smart
%and very specialised in this field -- huge understanding
%Can ask any questions, down-to-earth and very helpful. I could not ask for a better supervisor. 
%
%Chatzi is very approachable and kind, good at providing feedback at the intermediate sessions.
%


%I reached out to Ian from literally the other side of the world, when I was still an exchange student at Tsinghua University in Beijing. When no professors seemed to be interested in supervising me for my final capstone project, he was happy to do so. He even offered to do a Skype call with me to discuss the available options for the project. I am truly grateful for his generosity in offering to supervise me.
%Ever since coming to Bristol University for exchange, he arranged weekly meetings to discuss project updates, answer any troubling questions, suggest important litterature and provide general clarifications. In general, great academic guidance was received and time were always made available if additional questions arose. These weekly meetings were very helpful and kept me accountable throughout the entire process.
%I appreciate his welcoming character, his attentiveness, rigorous scholarship and his ability to clearly explain answers to my questions. He truly is an expert in the field of visualisation and pattern recogni- tion; a master of his craft. I am truly humbled by his expertise and I have great respect for Professor Nabney.


\newpage



%\include{abstract}
%\mainmatter
%\setcounter{tocdepth}{3}
%\setcounter{secnumdepth}{3} dunno what this does.
{
  \hypersetup{linkcolor=black}
  \tableofcontents
  % Two times compilation
  %\listoffigures
  %\listoftables
   
}
\newpage
%%%%%%%%%%%%%%%%%%%%%%%%%%%%%%%%%%%%%%%%%%%%%%%%%%%%%%%%%%%%%%%%%%%%%%%%%%%%%%%%%%%%%%%%%%%%%%%%
%%% Intro
%%%%%%%%%%%%%%%%%%%%%%%%%%%%%%%%%%%%%%%%%%%%%%%%%%%%%%%%%%%%%%%%%%%%%%%%%%%%%%%%%%%%%%%%%%%%%%%%
\chapter{Introduction}
Railway companies need to continuously and sufficiently maintain the train tracks and optimally detect defects in order to have a more punctual and more effective train system. However, the current track-condition assessment system is expensive, time consuming and ineffective, primarily involving manual inspection. This condition assessment is evolving from a manual-labor based approach to a data-driven focused industrialized assessment. This paradigm shift from manual to data-driven approach is taking place at this current point in time, as more railway companies are making this transition and funding research within this area as seen in \cite{Sharma2018DatadrivenOO}, \cite{modeltrain}, \cite{Deutsche21:online} and \cite{Howdigit41:online}.

Switzerland's \textit{Swiss Federal Railways} (SBB), is one of those companies being a part of this paradigm shift. SBB has specifically built one special \textit{Diagnostic Vehicle} (Diagnosefahrzeug - DFZ) designed for defect detection and identification among other purposes. For this, accelerometers have been installed at the front and back of the DFZ to collect the signal responses from the wheel and the train track (see appendix \ref{figs:veh} for the accelerometer positions).

A defect in train tracks can be seen as a discontinuity. As a train passes over this discontinuity, it will result in a perturbation that can be detected by sensors. It is our main assumption that each defect type will have a specific signature, which will allow for its identification and classification. This assumption is well-grounded in the litterature as seen in \cite{Introduc31:online}, \cite{pmid28590422} where the assumption is made for human activity recognition.

\section{Objective}
The objective of this project is to identify and classify rail surface defects. We aim to build an effective machine learning pipeline that takes information about defects as input and outputs a classification confidence for these defects. By succesfully identifying and classifying the defects, we take one step further toward predictive train-infrastucture maintenance, which will ultimately result in a more reliable network \cite{unknown}.

\section{Defects}
\label{sec:defects}
A defect can be seen as a deviation from a standard train track; and can be further sub-categorized based on official defect documentation \cite{rail:online} provided by \textit{International Union of Railways} (UIC). Upon inspection of defective tracks, SBB inspectors reports defects with reference to the UIC-based, defect report document by SBB. An example of this can be seen in \ref{app:report}.

Generally, a defect is separated into two overarching types: range- and point-defects. I.e. a defect that is detected at a single point versus a defect that is detected at varying lengths. A point defect is perceived as a sharp signal response, whereas a range-defect is perceived over a greater time period. Examples of point and range defects can be seen respectively in \ref{fig:pointdef} and \ref{fig:rangedef}.
\begin{figure}[H]
	
	\centering
	%\includegraphics[width = 0.49\textwidth]{imgs/defs/p(400,A-103206)0}
	%\caption{Defect ID: A-$103206$; Report set ID: $400$}
	\includegraphics[width = 0.2\textwidth]{imgs/defs/p(400,A-118152)0}
	\includegraphics[width = 0.38\textwidth]{imgs/defs/p(400,A-231400)0}
	\includegraphics[width = 0.38\textwidth]{imgs/defs/p(400,A-233599)0}
	\caption{\textbf{Left}: Schwelle: \textit{Beschädigte Betonschwellen} (A-$118152$); \textbf{middle:} Fahrbahn: \textit{Verletzungen} (A-$233599$); \textbf{right:} Vignolschiene: \textit{Verletzungen} (A-$231400$). The format is: track component: \textit{defect type} (defect ID). These have all been reported as defects (with subcategories) with a length equals to zero, and thus have been classified as point defects using our terminology. (Source of pictures: SBB's defect report document)}
	\label{fig:pointdef}
\end{figure}
\raggedbottom
\begin{figure}[H]
	\centering
	%\includegraphics[width = 0.49\textwidth]{imgs/defs/p(400,A-103206)0}
	%\caption{Defect ID: A-$103206$; Report set ID: $400$}
	\includegraphics[width = 0.29\textwidth]{imgs/defs/r(400,A-184063)0}
	\includegraphics[width = 0.29\textwidth]{imgs/defs/r(400,A-184063)1}
	
	\includegraphics[width = 0.29\textwidth]{imgs/defs/r(400,A-146358)0}
	\includegraphics[width = 0.29\textwidth]{imgs/defs/r(400,A-146358)1}
\end{figure}
\begin{figure}[H]
	\centering
	\includegraphics[width = 0.29\textwidth]{imgs/defs/r(400,A-103213)0}
	\caption{\textbf{Top:} Gleis-$149.8 m$: \textit{Head-Checking: Periodische Rissbildung / Ausbrechungen an der Fahrkante} (A-$184063$); \textbf{middle:} Schienenzwischenlage-$5.0 m$: \textit{Weiterer Abweichungstyp} (A-$146358$); \textbf{bottom}: Bankett-$1169.5 m$: \textit{Ungenügendes Schotterprofil} (A-$103213$). The format is: track component-length: \textit{defect type} (defect ID). These have all been reported as defects (with subcategories) with a length strictly greater than zero, and thus have been classified as range defects using our terminology. Most of these range defects have two pictures likely to give more detail. (Source of pictures: SBB's defect report document)}
	\label{fig:rangedef}
\end{figure}
\raggedbottom % to get of unnecessary vertical spacing
Furthermore, each defect is associated with a variety of attributes such as: associated component, defect component location, defect type etc. For this work, the focus has been on the defect component/category. That is, the track component with which the defect was identified on (see appendix \ref{app:defcomp}). This is, however, only a rough classification, as this attribute exists at a 'higher level', seeing that the lowest level of defect sub-division is the actual defect type (see appendix \ref{app:deftyp} for a list of defect types).

The defect attribute for analysis is not a trivial choice, as the signal responses likely depend on the defect type, its associated component and its location on the component etc. In the future, it will be important to account for other attributes. Albeit, this work only considers the defect components. In the following, 'defect' will thus refer to these defect components.

Finally, only the point defects are subject to analysis, as this simplifies the problem statement; range-defects and its associated factors are thus disregarded. In the future, range defects naturally also need to be taken into account.

\section{Switches and insulation joints}
To distinguish between defects and non-defects, this work also considers \textit{switches} and \textit{insulation joints}' seen in the figures \ref{fig:swi} and \ref{fig:ins}. Fact of the matter, is that, defect signals can vary a lot depending on the defect component. In addition, these sample amounts are far less compared to that of switches and insulation joints. With the incorporation of these non-defects, access to more samples is achieved, while these samples are suspected to have clearer signatures -- easier to classify.

\begin{figure}[H]
	\centering
	\includegraphics[width = 0.49\textwidth]{imgs/switch0}
	\includegraphics[width = 0.49\textwidth]{imgs/switch1}
	\caption{(Picture sources from right to left: \cite{swi0:online}, \cite{swi1:online})}
	\label{fig:swi}
\end{figure}
\begin{figure}[H]
	\centering
	\includegraphics[width = 0.49\textwidth]{imgs/ins_joint0}
	\includegraphics[width = 0.49\textwidth]{imgs/ins_joint1}
	\caption{(Picture sources from right to left: \cite{ins0:online}, \cite{ins1:online})}
	\label{fig:ins}
\end{figure}

In order to not get confused about the different terms, the word 'entity' will be used as an umbrella term for the different track entities: switch, insulation joint and defect.

\section{Data}
The data has been collected and provided by SBB. Using their DFZ, SBB has made trips back and forth to different cities in Switzerland in order to collect various data including but not limited to accelerometer data. After the data collection, the data is imported into a Python-VR object, which is composed of \verb|pandas| \verb|DataFrames|. \verb|Python| objects with pandas \verb|DataFrames| were chosen for their functional flexibility in terms of data storage and data accessibility.

The accelerometers capture the accelerations at the vertical (Z) and transversal (Y) axes (along with the timestamps at each recording), of which, only the Z axis was considered in this project, as it is assumed to be the most responsive to vertical perturbations. These accelerometers are installed on both leading (axle 1) and trailing axles (axle 4) of the measurement coach. See appendix \ref{figs:veh} for visualisations of the accelerometer placements on the DFZ. 

In this work, the following DFZ measurement rides in table \ref{tab:rides} are used for classification.

\begin{table}[H]
	\centering
	\begin{tabular}{|c|c|c|c|} \hline
		\textbf{From} & \textbf{To} & \textbf{Date} & \textbf{Campaign ID}\\ \hline \hline 
		Bern & Olten &  \verb|2019-05-27T08_55_55| & \verb|819Z DFZ01| \\ \hline 
		Olten & Bern &  \verb|2019-05-27T10_03_59| & \verb|077Z DFZ01| \\ \hline 
		Bern & Olten &  \verb|2019-05-27T13_05_53| & \verb|330Z DFZ01| \\ \hline 
		Olten & Bern &  \verb|2019-05-27T14_10_51| & \verb|425Z DFZ01| \\ \hline 
	\end{tabular}
	\caption{Measurement rides for classification. ('From' and 'To' are cities in Switzerland)}
	\label{tab:rides}
\end{table}

Lastly, defect attributes and locations were retrieved from SBB's database and synchronized with the measurement data.

\section{Code}
\label{int:sec:code}
The code is written purely in python. To create neural network architectures, \verb|keras| (version \verb|2.2.5|) along with \verb|tensorflow| (version \verb|2.1.0|) is used. \verb|keras| is essentially a high-level neural networks library which runs on top of \verb|tensorflow|. It has a consistent, simple API and provides clear and actionable feedback upon user error. Models are easily made by connecting configurable building blocks together, with few restrictions \cite{TensorFl31:online}. The models were trained in \verb|Google Colab|, which is a web application provided by Google that enables users to run python code in the web browser with access to GPUs\footnote{\url{https://colab.research.google.com/notebooks/intro.ipynb}}. It is very similar to \verb|Anaconda's| \verb|Jupyter Notebooks|, except that \verb|Colab| runs in the browser, is collaborative and provides free usage of GPUs (meaning model training goes faster).\\


\noindent All the code can be found on github: \\
\url{https://github.com/Aiyualive/SemesterProject2.0}.\\

\noindent The latest model evaluation workflow can be found in \verb|Colab|:\\
\url{https://colab.research.google.com/drive/10lxoYyX1bMRkdQiIQsf_87_xhXDccOKJ}

%keras version number, pandas version number, tensorflow version nmber


%%%%%%%%%%%%%%%%%%%%%%%%%%%%%%%%%%%%%%%%%%%%%%%%%%%%%%%%%%%%%%%%%%%%%%%%%%%%%%%%%%%%%%%%%%%%%%%%
%%% Design and implementation
%%%%%%%%%%%%%%%%%%%%%%%%%%%%%%%%%%%%%%%%%%%%%%%%%%%%%%%%%%%%%%%%%%%%%%%%%%%%%%%%%%%%%%%%%%%%%%%%
\chapter{Design and Implementation}
For the process of defect classification the pipeline in figure \ref{fig:pipeline} was designed. In the next sections, an overview will be given of how each step was implemented.
\begin{figure}[H]
	\centering
	\includegraphics[width = \textwidth]{imgs/pipeline}
	\caption{Primary pipeline}
	\label{fig:pipeline}
\end{figure}


%--> reevaluation| Realised that the results were not very good. So we need to visualise the data and look for separability to make a better model

\section{Localization of the DFZ}

The vehicle is localized using the combination of multiple data streams: track transponders (Eurobalise), GPS signals, Odometers and Switch detection systems. The location is reported with respect to the GPS Antenna of the vehicle which is close to the pantograph, where the reported localization error has been determined to be quite small. It is reported along with a timestamp corresponding to the location of the vehicle at that time, at the location of the antenna. 

\section{Shift of GPS timestamps}
Since the exact position (covered distance) at each accelerometer at either side of the GPS antenna is unknown, the goal in this step is to compute these from the reported location of the DFZ. The GPS sensor is sampled at a lower frequency compared to the accelerometers (every $25$ cm vs $24$ kHz respectively). From this, the corresponding GPS positions for each accelerometer sample necessarily need to be found. This is done by linear interpolation using the timestamps of the accelerometers and GPS. 

Depending on the direction of the vehicle, the offset between the accelerometers and the GPS sensor positions on the vehicle body is then added/subtracted to achieve the accelerometer positions. An illustration of this process is shown in \ref{fig:gpsshift}.

\begin{figure}[H]
	\centering
	\includegraphics[width = 0.7\textwidth]{imgs/gpsshift}
	\caption{The location of each timestamp of the accelerometers have to be determined through linear interpolation}
	\label{fig:gpsshift}
\end{figure}

\section{Peak windows}
Retrieving the signal response around the defect location forms a crucial aspect in the overarching pipeline. The goal of this step is to, around each defect, create a "window" containing accelerometer accelerations of a specified time length, wherein the highest acceleration recording around is found in the center. As a result, all of these windows would be uniform in the sense that they are all centered according to the highest recording of a defect. It is then assumed that each window forms the signature of each track entity. 

Since it is assumed that each track entity is identified by a well-formed peak, one first needs to find this peak within a reasonable offset from the defect location. Thereaftere, that peak is centered within a window with another reasonable window length offset.

In the code, this is done by defining two parameters: \verb|find_peak_offset = 1| and \verb|window_offset = 0.5|. I.e. given a defect timestamp, a search is performed to find the maximum, absolute acceleration recording that has occured $1$ second after and $1$ second before the defect timestamp. Once the peak has been found, it is centered in a $1$ second window ($0.5$ sec on each side). An illustration of this proces can be seen in \ref{fig:peak_windows}.

\begin{figure}[H]
	\centering
	\includegraphics[width = \textwidth]{imgs/peak_windows}
	\caption{Setting peak/acceleration windows for an entity}
	\label{fig:peak_windows}
\end{figure}

\section{Entity library}
The peak windows arguably forms the central feature of the defect library. However, based on domain knowledge, other features like vehicle speed as well as defect severity likewise needs to be considered for our neural network:

\begin{itemize}
	\item \textbf{Peak/acceleration windows:} accelerations of length 1 for each reported entity.
	\item \textbf{Vehicle speed (m/s):} vehicle speed at the closest timestamp of the reported entity.
	\item \textbf{Severity/urgency:} defects have an urgency label signifying how much time is left before acting; labels $1,2,3,4$, where urgency decreases in ascending order. Insulation joints and switches have been self-engineered with label $5$, as all entities need to have the same feature column for training.
\end{itemize}
Additional information about each entity has been retrieved as well, such as: timestamps, driving direction, corresponding entity IDs etc. 

Given a specific measurement ride object, we either retrieve each feature directly from the corresponding \verb|dataframe| or with the use of designated helper functions for those requring extra processing. Currently, have have a 2-level nested \verb|for loop|, looping for each axle outerly, and looping for each entity entry innerly.

The implementation of this could have made more elegant by operating directly on the dataframe, which might also increase speed of the implementation as the \verb|pandas| library has optimised their dataframe operations. However, speed and efficency was not a major concern in this project. 

The resulting plots for each entity can be seen in figures \ref{fig:insjoint}, \ref{fig:switches}, \ref{fig:defects}; for the defects seen in the the introduction, refer to appendix \ref{app:introdefects}.
%refer to the defects presented in introduction?,  appendix for more signals?
\begin{figure}[H]
	\centering
	\includegraphics[width = 0.49\textwidth]{../AiyuDocs/new_signalplots/insulationjoint_6_AXLE_11_077Z_DFZ01}
	\includegraphics[width = 0.49\textwidth]{../AiyuDocs/new_signalplots/insulationjoint_6_AXLE_12_077Z_DFZ01}
	
	\includegraphics[width = 0.49\textwidth]{../AiyuDocs/new_signalplots/insulationjoint_6_AXLE_41_077Z_DFZ01}
	\includegraphics[width = 0.49\textwidth]{../AiyuDocs/new_signalplots/insulationjoint_6_AXLE_42_077Z_DFZ01}
	\caption{Insulation Joint}
	\label{fig:insjoint}
\end{figure}

\begin{figure}[H]
	\centering
	\includegraphics[width = 0.49\textwidth]{../AiyuDocs/new_signalplots/switches_46_AXLE_11_425Z_DFZ01}
	\includegraphics[width = 0.49\textwidth]{../AiyuDocs/new_signalplots/switches_46_AXLE_12_425Z_DFZ01}
	
	\includegraphics[width = 0.49\textwidth]{../AiyuDocs/new_signalplots/switches_46_AXLE_41_425Z_DFZ01}
	\includegraphics[width = 0.49\textwidth]{../AiyuDocs/new_signalplots/switches_46_AXLE_42_425Z_DFZ01}
	\caption{Switches}
	\label{fig:switches}
\end{figure}

\begin{figure}[H]
	\centering
	\includegraphics[width = 0.49\textwidth]{../AiyuDocs/new_signalplots/Vignolschiene_8_AXLE_11_819Z_DFZ01}
	\includegraphics[width = 0.49\textwidth]{../AiyuDocs/new_signalplots/Vignolschiene_8_AXLE_12_819Z_DFZ01}
	
	\includegraphics[width = 0.49\textwidth]{../AiyuDocs/new_signalplots/Vignolschiene_8_AXLE_41_819Z_DFZ01}
	\includegraphics[width = 0.49\textwidth]{../AiyuDocs/new_signalplots/Vignolschiene_8_AXLE_42_819Z_DFZ01}
	\caption{Defect: vignolschiene}
	\label{fig:defects}
\end{figure}


\section{Classification}
A primary \verb|NN| (neural network) class along with a \verb|ModelMaker| class have been created. The former does everything from pre-processing the data to evaluating the used model. The latter, as the name suggests, is utlised for creating and using different models, which is useful as one can keep track of how the models have been modified and improved. 

\subsection{NN class}
To make a classification, relevant feature selection needs to be performed. The features are then simply fed into an \verb|NN| object, where the API of the \verb|NN| class can be called for classification. The usage of the \verb|NN| class is demonstrated below in \ref{tab:nnclass}.

\begin{table}[H]
	\centering
	\small
	\begin{tabular}{l p{0.5\linewidth}}\noalign{\global\arrayrulewidth=0.3mm} 
	\hline 
	\multicolumn{2}{c}{\textbf{API of NN class}}
		\\ \hline  
		\noalign{\global\arrayrulewidth=0.05mm}
		\verb|__init__()|                  & initialises a \verb|NN| object\\ \hline
		\verb|prepare_data()|            & pre-process data, this includes standardisation of data\\ \hline
		\verb|make_model()|              & uses \verb|ModelMaker| class to select a model \\ \hline
		
		\verb|fit()|			         & trains the model\\ \hline
		\verb|classify()|			     & uses the trained model to classify on a test set\\ 
		\noalign{\global\arrayrulewidth=0.3mm} \hline 

		\vspace*{0.25cm}\\
		\hline
		\multicolumn{2}{c}{ \textbf{Other utility API functions} }	
		\\ \hline 
		\noalign{\global\arrayrulewidth=0.05mm}
		\verb|measure_performance()| &  \\ \hline
		\verb|plot_metrics()|          \\  \hline
		\verb|plot_confusion_matrix()| \\ \hline
		\verb|load_weights()|             \\ \hline
		\verb|load_model_()|             \\ \hline
		\verb|save_history()|           \\ \hline
		\verb|save_model()|              \\ \hline
		\verb|save_classification_to_csv()|  \\ \hline
		\noalign{\global\arrayrulewidth=0.3mm}
		\hline \hline
		\verb|run_experiment()| & evaluates the given model for a \# of repetitions\\ 
		\noalign{\global\arrayrulewidth=0.3mm}
		\hline
	\end{tabular}
	\caption{To train a model, the first API functions needs to be called sequentially. Other utility functions are rather self-explanatory.}\label{tab:nnclass}
\end{table}
\raggedbottom % To get rid of weird spacing

\subsection{ModelMaker class}
\label{sec:modelmaker}
Below is the code for building the baseline model in \ref{sec:modeval} using \verb|keras|. 
\begin{minted}[fontsize=\footnotesize, frame=single, linenos=true]{python}
def _makeModel2(self):
    model = Sequential()
    model.add(Reshape((self.seq_len, self.channels),
                       input_shape=(self.seq_len * self.channels,)))
    model.add(Conv1D(filters=20,
                     kernel_size=10,
                     input_shape=(self.seq_len, self.channels),
                     activation= 'relu')) 
    model.add(MaxPooling1D(pool_size=24))
    model.add(Flatten())
    model.add(Dense(16, activation='relu'))
    model.add(Dropout(rate=0.3,seed=1))
    model.add(Dense(self.n_classes, activation='softmax'))
    model.compile(loss='categorical_crossentropy',
                  optimizer=Adam(learning_rate=0.0001),
                  metrics=self.metrics)
    self.model = model
\end{minted}

\noindent \textbf{Line 1-4}\\
The model is firstly defined as a feed forward neural network. Secondly, the input shape has to be correctly formatted. As input, the model needs to receive the input shape of (number of samples, signal length $\times$ number of channels). Each sample is then converted into the shape: (signal length, number of channel). For instance, inputting 4 axle channels each of $24000$ (24 kHZ accelerometers) acceleration readings is converted: $(24000 \times 4,) \rightarrow (24000, 4)$\\

\noindent \textbf{Line 5-8}\\
The input is then convolved with $20$ fitlers/feature detectors of kernel size $10$. \verb|relu| is the Rectified Linear Activation, which is the standard activation function used in the majority of machine learning applications. Insert drawing\todo{}\\

\noindent\textbf{Line 9}\\
For each filter, \verb|MaxPooling| takes a number of accelerations defined by the \verb|pool_size| as input and outputs the max acceleration, resulting in a smaller signal of length: original signal length divided by \verb|pool_size| . In the general case, this would correspond to segmenting the filters into retrieving the max accelerations. Insert picture\todo{}\\

\noindent \textbf{Line 10-13} \\
The resulting filters of the max pooling is then flattened into a 1D array, after which it is fed into 16 neurons with \verb|relu| activation. A \verb|Dropout| layer is then added to prevent overfitting; by randomly removing neuron connections. Finally, each neuron in the last \verb|Dense| layer outputs a probability (using \verb|softmax| activation) corresponding to its designated class. The final classification class of a training sample is then taken to be the maximum of these probabilities. Insert picture \todo{}.\\

\noindent \textbf{Line 14-17} \\
The model is then compiled, where the overall training goal is to minimize the cost/loss function, \verb|cross-entropy|, which is the common loss function used for multi-class classification.

\section{Feature-set visualisations}
After evaluating the results from the model (results can be seen in \ref{sec:modeval}), we have not achieved any significant results. Using PCA on the feature-set, the aim is to visualise the class clustering.\todo{}

%%%%%%%%%%%%%%%%%%%%%%%%%%%%%%%%%%%%%%%%%%%%%%%%%%%%%%%%%%%%%%%%%%%%%%%%%%%%%%%%%%%%%%%%%%%%%%%%
%%% Evaluation
%%%%%%%%%%%%%%%%%%%%%%%%%%%%%%%%%%%%%%%%%%%%%%%%%%%%%%%%%%%%%%%%%%%%%%%%%%%%%%%%%%%%%%%%%%%%%%%%
\chapter{Evaluation}

In this chapter, the results are presented and the findings are discussed. Insert more describing text? More effort is need in general for this chapter\todo{}

\section{Baseline Model}
\label{sec:model}
Experimenting with models have been an ongoing process throughout this project; in the end a final baseline model has been determined to be the one seen in table \ref{tab:model}. This very simple convolutional neural network  is explained previously in \ref{sec:modelmaker}. The hyper-parameters that are used for training can be seen in table \ref{tab:hyperp}.
\begin{table}[H]
	\centering
	\begin{tabular}{|c|c|c| c|} \hline
		\textbf{Layer} & \textbf{Output Shape } & \textbf{Number of params} \\ \hline \hline 
		Conv1D & $(, 23991, 20)$  &  220  \\ \hline 
		MaxPooling (24) & $(, 999, 20)$  &  -  \\ \hline 
		Flatten & $(, 19980)$  &  -  \\ \hline 
		Dense & $(, 16)$  &  319696  \\ \hline 
		Dropout & $(, 16)$  &  -  \\ \hline 
		Dense & (, num\_classes)  &  187  \\ \hline 
	\end{tabular}
	\caption{Baseline model}
	\label{tab:model}
\end{table}

\begin{table}[H]
	\centering
	\begin{tabular}{|l|c|} \hline
		\textbf{Hyper-parameters} & \textbf{Value} \\ \hline \hline 
		Epochs  & $20$ \\ \hline 
		Batch size & Full training set  \\ \hline 
		Optimizer & Adam ($0.0001$)  \\ \hline 
		Loss function & Cross-entropy  \\ \hline 
	\end{tabular}
	\caption{Hyperparameters and its accompanying values}
	\label{tab:hyperp}
\end{table}
The epoch value was found to strike a good balance for the bias-variance tradeoff (basically prevent overfitting to the training data). Since memory is not of high importance, we input the full training set as the batch size. Meaning, all the training samples are seen at once for each training epoch. The chosen optimizer is the default choice in the Deep Learning community and provides adaptive learning rate with initial value set to $0.0001$. Finally, the cross-entropy loss function is commonly used for multi-class, categorical classification. Insert drawing of model\todo{}

\section{Dataset class distributions}
Given the full feature-set, the model is trained on $85\%$ of the data and validated on the remaining $15\%$ both seen in \ref{tab:classdist}. The model is trained, crucially, by weighing the imbalanced classes according to its percentage of the total training set. That is, minority classes are given higher importance.

Worth mentioning is the fact that this is the same validation set that was used for model selection during training, which is not best practice. It would have been ideal to use a separate un-seen test set for this evaluation, although with the limited training data, it has been deemed adequate. Otherwise not enough information would be sufficient for both training and evaluation. 
%The breakdown of the training an validation set can be found in table \ref{tab:classdist}.
\begin{table}[H]
	\centering
	\begin{tabular}{|r|l|}
	\multicolumn{1}{c}{}\\ \hline
	\textbf{Class} & \textbf{Defect Comp.} \\ \hline
		0  & Fahrbahn \\ \hline 
		1  & Gleis \\ \hline 
		2  & Herzstück \\ \hline 
		3  & Schiene \\ \hline 
		4  & Schienenzwischenlage \\ \hline 
		5  & Schotter \\ \hline 
		6  & Schwelle \\ \hline
		7  & Vignolschiene \\ \hline
		8  & Zungenschiene \\ \hline
		9  & Ins. joint  \\ \hline
		10 & Switch \\ \hline
	\end{tabular}
	\begin{tabular}{|c|c|c|} \hline
	\multicolumn{2}{c}{\textbf{Training set}}\\ \hline
	\makebox[4em]{\textbf{Count}} & \makebox[4em]{\textbf{Percent}} \\ \hline
		$93$  & $4.02\%$ \\ \hline 
		$57$  & $2.46\%$ \\ \hline 
		$46$  & $1.99\%$ \\ \hline 
		$180$ & $7.78\%$ \\ \hline 
		$50$  & $2.16\%$ \\ \hline
		$148$ & $6.39\%$ \\ \hline 
		$51$  & $2.20\%$ \\ \hline
		$161$ & $6.95\%$ \\ \hline
		$34$  & $1.47\%$ \\ \hline
		$988$ & $42.68\%$ \\ \hline
		$507$ & $21.90\%$ \\ \hline
	\end{tabular}
%	\vspace*{0.5cm}\\
	\begin{tabular}{|c|c|c|} \hline
	\multicolumn{2}{c}{\textbf{Validation set}}\\ \hline 
	\makebox[4em]{\textbf{Count}} & \makebox[4em]{\textbf{Percent}} \\ \hline
		$11$  & $2.69\%$ \\ \hline 
		$15$  & $3.67\%$ \\ \hline 
		$10$  & $2.44\%$ \\ \hline 
		$32$  & $7.82\%$ \\ \hline 
		$6$   & $1.47\%$ \\ \hline 
		$20$  & $4.89\%$ \\ \hline 
		$13$  & $3.18\%$ \\ \hline
		$35$  & $8.56\%$ \\ \hline
		$10$  & $2.44\%$ \\ \hline
		$160$ & $39.12\%$ \\ \hline
		$97$  & $23.72\%$ \\ \hline
	\end{tabular}
	\caption{Class distributions among training and validation set. The full dataset consists of $\bm{2724}$ samples. The training set totals $\bm{2315}$ samples, and the validation set totals $\bm{409}$ samples.}
	\label{tab:classdist}
\end{table}

Worth mentioning is the fact that each of these were trained on different training and validation sets. This was a mistake that was only realised later, where the ideal would be to use the same sets for all the model and feature set combinations. For now, the differences are rather minor and \ref{tab:classdist} shows a sample distribution for evaluation of the final  featureset (row) in the metric results table, \ref{tab:metresult}. 

%\newpage
\section{Metrics and feature-set}
As the training process is stochastic in nature, the model is trained 10 times (experiments), after which an average along with the associated standard deviation are computed. In addition, the following metrics are also used to define the model performance: loss (\textbf{L}), accuracy (\textbf{ACC}), F1-score (\textbf{F1}), area under the curve (\textbf{AUC}) and the associated confusion matrix (\textbf{CM}). \todo{explain metrics?} These results are summarised in table \ref{tab:metresult}.

The baseline model has been utilised to determine the best feature-set with the assumption that this feature set would generalize to future NN models. The feature-set combinations can be found in table \ref{tab:features}.

\begin{table}[H]
	\centering
	\begin{tabular}{|l|c|l|} \hline
		\textbf{Feature-sets} & \textbf{\# feat.} & \textbf{Parameters} \\ \hline \hline 
		All-AX    & 4 & 1s acc. window \\ \hline 
		AX        & 1 & 1s acc. window \\ \hline 
		AX-LP     & 1 & 1s acc. window filtered at $0.63$ threshold \\ \hline
		AX,SP     & 2 & 1s acc. window, vehicle speed \\ \hline
		AX-LP, SP & 2 & 1s acc. window filtered at $0.63$ threshold, vehicle speed \\ \hline 
	\end{tabular}
	\caption{All feature-sets used for defect classification. \textbf{Reminder}: 1 second window corresponds to 24000 samples since each accelerometer is sampled at 24kHz}
	\label{tab:features}
\end{table}

% 0.5 windows
%Overall loss: 1.847% (+/-0.130)
%Overall ACC: 0.440% (+/-0.015)
%Overall AUC: 0.811% (+/-0.049)
%Overall F1: 0.346% (+/-0.024)
%Overall F1W: 0.247% (+/-0.054)
%Overall AVG_AUC: 0.578% (+/-0.048)
\begin{table}[H]
		\centering 
		\begin{tabular}{|l||*{5}{c|}}\hline
			\backslashbox{Metric}{Model}
			&\makebox[5em]{L}&\makebox[5em]{ACC)}&\makebox[5em]{F1}&\makebox[5em]{AUC} 
			\\\hline\hline
			BL (ALL-AX)   & $2.089\ (\pm 0.118)$ & $0.411\ (\pm 0.144)$ & $0.303\ (\pm 0.107)$ & $0.508\ (\pm 0.048)$ \\ \hline
			BL (AX) & $2.223\ (\pm 0.050)$ & $0.378\ (\pm 0.041)$ & $0.271\ (\pm 0.031)$ & $0.463\ (\pm 0.060)$ \\ \hline
			BL (AX-LP) & $2.223\ (\pm 0.069)$ & $0.384\ (\pm 0.034)$ & $0.311\ (\pm 0.030)$ & $0.540\ (\pm 0.032)$ \\ \hline
			BL (AX, SP) & $1.850\ (\pm 0.112)$ & $0.431\ (\pm 0.025)$ & $0.337\ (\pm 0.027)$ & $0.583\ (\pm 0.054)$ \\ \hline
			BL (AX-LP, SP) & $1.845\ (\pm 0.118)$ & $0.443\ (\pm 0.021)$ & $0.357\ (\pm 0.026)$ & $0.593\ (\pm 0.052)$ \\ \hline
		\end{tabular}
		\captionof{table}{Average metrics times and their standard deviations in parenthesis over 10 experiments. BL is short for baseline model. The parenthesis denotes the used feature-set. (All-AX), (AX), (AX, LP) and (AX,SP) denote that a single training sample respectively consists of: all 4 axle channels, 1 individual axle channel, 1 individual axle channel low-pass filtered and 1 individual axle channel along with its corresponding speed.}
		\label{tab:metresult}
\end{table} 

\section{Plots}
In the following, the training and validation plots for each metric (loss, acc) during each epoch are presented. The format is: \textbf{Top:} confusion matrix for the predicted vs true classes/labels. \textbf{Left:} average training/validation loss accompanied by error bars. \textbf{Right}: average training/validation accuracy accompanied by error bars.

\begin{figure}[H]
	\centering
	\includegraphics[width = 0.39\textwidth]{models/PLOTS/axall-cm}
	
	\includegraphics[width = 0.39\textwidth]{models/PLOTS/axall-avg-loss}
	\includegraphics[width = 0.39\textwidth]{models/PLOTS/axall-avg-acc}
	\caption{\textbf{BL (ALL-AX)}}
\end{figure}

\begin{figure}[H]
	\centering
	\includegraphics[width = 0.39\textwidth]{models/PLOTS/ax-cm}
	
	\includegraphics[width = 0.39\textwidth]{models/PLOTS/ax-avg-loss}
	\includegraphics[width = 0.39\textwidth]{models/PLOTS/ax-avg-acc}
	\caption{\textbf{BL (AX)}}
\end{figure}

\begin{figure}[H]
	\centering
	\includegraphics[width = 0.39\textwidth]{models/PLOTS/axlp-cm}
	
	\includegraphics[width = 0.39\textwidth]{models/PLOTS/axlp-avg-loss}
	\includegraphics[width = 0.39\textwidth]{models/PLOTS/axlp-avg-acc}
	\caption{\textbf{BL (AX-LP)}}
\end{figure}

\begin{figure}[H]
	\centering
	\includegraphics[width = 0.39\textwidth]{models/PLOTS/axsp-cm}
	
	\includegraphics[width = 0.39\textwidth]{models/PLOTS/axsp-avg-loss}
	\includegraphics[width = 0.39\textwidth]{models/PLOTS/axsp-avg-acc}
	\caption{\textbf{BL (AX, SP)}}
\end{figure}

\begin{figure}[H]
	\centering
	\includegraphics[width = 0.39\textwidth]{models/PLOTS/axlpsp-cm}
	
	\includegraphics[width = 0.39\textwidth]{models/PLOTS/axlpsp-avg-loss}
	\includegraphics[width = 0.39\textwidth]{models/PLOTS/axlpsp-avg-acc}
	\caption{\textbf{BL (AX-LP, SP)}}
\end{figure}

The remaining metric plots could not be retrieved as \verb|Tensorflow| has to be configured in quite a complex manner to make it account for other metrics. This could however be incorporated with more time. Albeit,he primary utility of metrics during training is for logging and for utilising suitable callbacks.\\

Add individual training plots for each experiment to the appendix. These are only confusion matrices taken from one experiment, although it would be better to compute the average confusion matrix from all the experiments. Perhaps, it would be better to just keep the last result plot and move the worse results to the appendix\todo{}

\section{Visualisation of class clustering}
\begin{figure}[H]
	\centering	
	\includegraphics[width = 0.5\textwidth]{imgs/class_sep0}
	
	\includegraphics[width = 0.5\textwidth]{imgs/class_sep1}
	\caption{PCA class separability}
\end{figure}

Insert more plots like this, has not been done fully\todo{}

\newpage
\section{Discussion}
From the resulting metrics and plots, it is evident that no truly significant results were achieved. However it can be said with good confidence that using a feature-set consisting of lowpass-filtered acceleration windows in addition to vehicle speed results in better performance. These findings suggests that further pre-processing techniques and more sophisticated feature selection will achieve better classification results. 

With regard to each feature-set, it can be seen from \textbf{all 4 channels}, that it has the worst confidences as the standard deviations are rather high compared to the other feature-sets. The accompanying plots also illustrate this matter. For this particular case, this can likely be explained by the four times smaller sample amount compared to the other feature-sets, as all 4 axle channels are combined into one sample. Minority-class removal has been done at the same threshold as the others (class samples above 10). Generally, it can be said that the model's poor results are very likely attributed to the small-size and class imbalance of the dataset.

For \textbf{single axle}, actual confidence in results can be seen by the decrease in error bars in the plots. Although, overfitting to the training data happens after the 14th epoch, as the training loss dips below the validation loss. To prevent this from happening, one can employ \verb|callbacks| during training such that it stops when meeting certain user-defined requirements. For this evaluation, this has not been done as variables are kept the same throughout feature-set evaluations.

\textbf{Low-pass} filtering the acceleration windows generally follows the same trend as the previous feature-set. For this case it can be seen that this extra pre-processing step improves performance of metrics. When accounting for the \textbf{single axle channel with vehicle speed}, it can be seen that more classes can be predicted; resulting in more of a general classifier as well as achieveing higher overall metric scores. For this particular feature-set (and the next) it can be seen that there is still a bit more room for improvement in the training loss as it is still in the process of converging. 

The last examined feature-set, \textbf{single, low-pass filtered axle channel and corresponding vehicle speed}, has the best performance metrics out of all and shows better variety in its classification predictions.

Furthermore, the confusion matrices show that there is no prediction for some of the classes. This is likely due to the segmentation of the signal when passing the input through the model as explained in \ref{sec:modelmaker}. It is suspected that a prediction for each class could be reached by perhaps \verb|average pooling| instead of \verb|max pooling|. 

From all confusion matrices, it can be observed that preference is given to a select-few classes. Namely, number 9 (insulation joint) as expected, number 3, 7, 5. Surprisingly enough, for the switches (number 10) few predictions exist and they primarily gets classified as insulation joints. In fact, from looking at the acceleration windows of switches, they do not seem distinct from any other signal. This is however in contrast to the insulation joints, where clearly defined maximal peaks are defined at the center. 

Finally, other important factors to take into account when evaluating this model, are the age of the tracks and severity of the defects. The utilised data in this work are produced from the DFZ which drove on brand new tracks built in 2006-2018. As a result, the defects severities are all minor; $648$ and $304$ out of $1264$ defects, are labeled 4 and 3 respectively (lowest severity labels).

Comment more on the metrics individually in the future.\todo{}
%%%%%%%%%%%%%%%%%%%%%%%%%%%%%%%%%%%%%%%%%%%%%%%%%%%%%%%%%%%%%%%%%%%%%%%%%%%%%%%%%%%%%%%%%%%%%%%%
%%% Conclusion and future work
%%%%%%%%%%%%%%%%%%%%%%%%%%%%%%%%%%%%%%%%%%%%%%%%%%%%%%%%%%%%%%%%%%%%%%%%%%%%%%%%%%%%%%%%%%%%%%%%
\chapter{Future work and conclusion}
This future work section sounds more like a discussion and perhaps needs to be moved to the previous chapter\todo{}
%%% FUTURE WORK
\section{Future work}
There is still a lot of work that needs to be done in this project. These are summarised in the next sub-sections, in no particular order. 

\subsection{Channels}
It might be interesting to consider sensor fusion of the ZY axes and the axle channels (8 total channels). The transverse axis, Y, would especially be valuable for switches as this channel can pick up sideways hits in the case of switches. These switch signals are clearly different from that of the other entities and may thus be more easily separable.
	
In this project, only the accelerometer accelerations at the axle-boxes were considered. However, the other accelerometer sensors higher up in the vehicle: bogie(Z, Y) and body (Y) (see appendix \ref{figs:veh}), will also be worth analyzing. 

Finally, in order to ensure that each axle channel response is uniform, it is also important to account for the varying calibrations of the accelerometers.

\subsection{Feature-set}
Another interesting aspect would be to account for the severity as a feature/input to the model; or even as the classification output.

For periodic, heartbeat electrocardiogram signals, heartbeat-specific features have been extracted as seen in \cite{10.3389/fphy.2019.00103}. Perhaps, a similar approach could have been employed for this project. Although, the entity signals are not periodic, and may thus not have the same applicability.

\subsection{Peak windows}
If the peak windows are not retrieved, sized and positioned optimally, much information about the entities are lost. Hence, the acceleration windows arguably forms the most crucial aspect of the feature-set. Evidently, the goal of these windows is to strike a fine balance between capturing the most important information vs. capturing disruptive information. E.g. another peak at the corners of the window or merged peaks. Thus, the ideal would be to search for the peaks within an appropriate offset from the defect location and then center around the 'best' peak with the smallest possible window. For this project, a window length of 1 second was set for all entities. Although, a smaller window length could likely have been employed.

Furthermore, each entity vary a lot in terms of the signal responses. E.g. switch signals usually have two high amplitude responses. For this, it might be suitable to define entity specifc window lengths. This would however necessitate a slightly more sophisticated neural network, as the inputted signal lengths would not be equal, thus requiring multiple inputs. 

\subsection{Identification of entities}
The identification of the infrastructure components has relied on a direct information retrieval from SBB's databases. Albeit, they are not always reliable. To improve the identification method, a fusion of different track entity information would provide a better measure of identification. I.e. the order and distance between entities, as well as the uncertainty in the reported location of the DFZ. Considering that insulation joints and switches might overlap in the signal responses, this would be an improvement to the current identification process. A visualisation of this could be seen in listing \ref{layout}. S, I, C and '...' (dots) denote switch, insulation joint, distance between track component respectively.

\begin{lstlisting}[frame=single, caption=Expected layout of entities vs actual layout., label=layout]
    Expected: "...I,....S,I,....C,....I,..............I,...S,..C,.."
    Measured: "...I,..S,...I,...C,...I,......I,...,...S,..C,.."
\end{lstlisting}

Furthermore, at times the defect can only be detected from one side of the track (AX11 and AX12) as one defect crossing sample with two signatures seen from different location. This would help in cases where the defect is only on one side of the track.

\subsection{Class separability}
More emphasis should be placed on feature-set class separability to determine which set of features provide the best classification confidence; better class clustering leads to better classification.\todo{}

The visualisation of class separation should arguably have been done first. Considering the fact that if a clear cluster separation exists, then applying a neural network would be a bit exaggerated. In that case, a simple multi-class support-vector machine from the \verb|sklearn| library would suffice for this task.

\subsection{Visualisation of neural network}
The idea of this subsection is to use class activation maps to visualise how the neural network makes its decisions.\todo{}

\subsection{Real-world simulated defects}
A crucial limitation to this project is the defects sample amounts. Another approach to gather these samples is to self-engineer them by implementing a small-scale model of the real-life train infrastructure. This was exactly the goal for our team in the ETH Hatchery \footnote{\url{https://sph.ethz.ch/eth-week-hatchery/}}, where we built such a model to collect signal responses from self-engineered defects. Even though it would be an extremely simplified model of the real-world, it would, however, provide proof of conccept for the idea of data-driven defect identification and classification.

%%% CONCLUSION
\newpage
\section{Conclusion}
The aim of this project has been to identify and classify rail surface defects. A machine learning pipeline that takes information about defects as input and outputs the corresponding classification confidence, has been built. The empirical space of focus has been on track defect components of zero-length, where the identification of these has relied solely the information provided in SBB's database.

The resulting classification has not shown any significant results. They have shown that using the retrieved acceleration windows are usable for this classification task, although there is much room for improvement by potentially tweaking the model as well as working with other features. It has been found that inputting the low-pass filtered acceleration windows along with corresponding vehicle speed produces notable improvement in the classification confidence

Despite the mediocre results, this work lays a good foundation to move onwards with further research. Not to mention the fact the library code is built in a modular, extensible manner for further improvements.

\newpage
\section{Todo}
\begin{itemize}
	\item insert pictures
	\item make the training sets uniform hroughout all the model/feature evaluations
	\item see if outlier removal helps?
	\item Do class separation visualisations
	\item test the auc score
	\item Change name of SBBproject to feature generator
	\item \verb|defects['severity'].value_counts()|
	\item Overall model explanation - poolsize of 24, 20 filters
\end{itemize}
%%%%%%%%%%%%%%%%%%%%%%%%%%%%%%%%%%%%%%%%%%%%%%%%%%%%%%%%%%%%%%%%%%%%%%%%%%%%%%%%%%%%%%%%%%%%%%%%
%%% Bibliography
%%%%%%%%%%%%%%%%%%%%%%%%%%%%%%%%%%%%%%%%%%%%%%%%%%%%%%%%%%%%%%%%%%%%%%%%%%%%%%%%%%%%%%%%%%%%%%%
\bibliography{master}
%more would be good. and not only blogposts. 
%--> Machine Learning and Pattern Recognition book (the one that is probably also used inthe Deep learning class that you took).
%--> Railway-vehicle dynamics by Knothe,
%--> UIC defect catalog which SBB uses as a base for classifying the defects
%--> Some scientific articles on sound recognition/voice

%You can put more citations, for example about some articles focusing on recognizing speech/sound, signals which have also very similar signatures
\newpage
\cleardoublepage
\appendix
%%%%%%%%%%%%%%%%%%%%%%%%%%%%%%%%%%%%%%%%%%%%%%%%%%%%%%%%%%%%%%%%%%%%%%%%%%%%%%%%%%%%%%%%%%%%%%%%
%%% Appendix introduction
%%%%%%%%%%%%%%%%%%%%%%%%%%%%%%%%%%%%%%%%%%%%%%%%%%%%%%%%%%%%%%%%%%%%%%%%%%%%%%%%%%%%%%%%%%%%%%%

\chapter{Introduction}
\section{List of defect components}
\label{app:defcomp}
\begin{table}[H]
	\centering
	\begin{tabular}{|l|c|} \hline
		\textbf{Bankett} \\ \hline 
		\textbf{Befestigung} \\ \hline
		\textbf{Dienstweg} \\ \hline
		\textbf{EKW / DKW} \\ \hline
		\textbf{Einfache und symmetrische Weiche} \\ \hline
		\textbf{Entwässerungsgraben} \\ \hline
		\textbf{Fahrbahn} \\ \hline
		\textbf{Gleis} \\ \hline
		\textbf{Gleisbettung} \\ \hline
		\textbf{Halbe Zungenvorrichtung} \\ \hline
		\textbf{Herzstück} \\ \hline
		\textbf{Herzstückspitze} \\ \hline
		\textbf{Kleber} \\ \hline
		\textbf{Rippenplatte} \\ \hline
		\textbf{Schiene} \\ \hline
		\textbf{Schienenklemme} \\ \hline
		\textbf{Schienenzwischenlage} \\ \hline
		\textbf{Schotter} \\ \hline
		\textbf{Schweissverbindung} \\ \hline
		\textbf{Schwelle} \\ \hline
		\textbf{Stockschiene} \\ \hline
		\textbf{Vignolschiene} \\ \hline
		\textbf{Weiche} \\ \hline
		\textbf{Zungenschiene} \\ \hline
	\end{tabular}
	\hspace*{0.5cm}	
	\begin{tabular}{|l|c|} \hline
		\textbf{Insulation Joint} \\ \hline
		\textbf{Switches} \\ \hline 
	\end{tabular}
	\caption{The entities that are subject to classification}
\end{table}

\section{List of defect types}
\label{app:deftyp}
\begin{table}[H]
	\centering
	\begin{tabular}{|l|} \hline
		\textbf{AT-Schweissung mit Squat}\\ \hline
		\textbf{Ausbruch in der Zunge} \\ \hline
		\textbf{Bankett}\\ \hline
		\textbf{Beschädigte Betonschwellen} \\ \hline 
		\textbf{Entwässerung oder Wasserkanal verstopft}\\ \hline
		\textbf{Gleisgeometrie} \\ \hline
		\textbf{Grundparameter}\\ \hline
		\textbf{Gleisgeometrie, Spurweite} \\ \hline
		\textbf{Gleisgeometrie, Standardabweichungen}\\ \hline
		\textbf{Head-Checking:  Periodische Rissbildung} \\
		\textbf{/ Ausbrechungen an der Fahrkante} \\ \hline 
		\textbf{Isolierstoss mit Überwalzung}\\ \hline
		\textbf{Kilometer-, Hektometer- oder Metertafeln} \\ \hline
		\textbf{Lose/fehlende Schienenbefestigung}\\ \hline
		\textbf{Lose/fehlende Schwellenbefestigung}\\ \hline
		\textbf{Mitten in der Schienenlänge: Oberflächenfehler}\\ \hline
		\textbf{Mitten in der Schienenlänge: Schleuderstellen oder Schnellbremsspuren}\\ \hline
		\textbf{Mitten in der Schienenlänge: Squat} \\
		\textbf{/ Rissbildung und örtliche Einsenkung der Lauffläche}\\ \hline
		\textbf{Oberflächlicher Fehler (Flügelschiene oder Herz)}\\ \hline
		\textbf{Schienenende: Sprödbruch}\\ \hline
		\textbf{Schienenende: Verquetschung}\\ \hline
		\textbf{Schienenende: örtliche Einsenkung der Lauffläche}\\ \hline
		\textbf{Schotter auf Schwellen}\\ \hline
		\textbf{Schotterfliessen, weisser Schotter}\\ \hline
		\textbf{Ungenügendes Schotterprofil}\\ \hline
		\textbf{Verletzungen}\\ \hline
		\textbf{Weiterer Abweichungstyp}\\ \hline
		\textbf{Zungenspitze angefahren}\\ \hline
	\end{tabular}
	\caption{Defect types}
\end{table}
\section{Vehicle and accelerometer placements}
\label{figs:veh}
\begin{figure}[H]
	\centering
	\includegraphics[width = 0.58\textwidth]{imgs/veh0}
	\label{fig:veh0}
	
	\includegraphics[width = 0.58\textwidth]{imgs/veh1}
	\label{fig:veh1}
	
	\includegraphics[width = 0.58\textwidth]{imgs/veh2}
	\caption{Illustrations of the accelerometer placements on the DFZ. For this project we have only considered axle number 4 in the third figure. (Source of picture: SBB documents)}
	\label{fig:veh2}
\end{figure}

\section{SBB defect report example}
\label{app:report}
\begin{figure}[H]
	\centering
	\includegraphics[width = 0.9\textwidth]{imgs/defs/rep0}
	\caption{}
	\includegraphics[width = 0.9\textwidth]{imgs/defs/rep1}
	\caption{A typical report for an arbitrary defect usually contains one description page followed by its picture(s)}
	\label{fig:veh2}
\end{figure}

\section{Defects in Introduction}
\label{app:introdefects}
\begin{figure}[H]
	\centering
	\includegraphics[width = 0.49\textwidth]{../AiyuDocs/new_signalplots/Schwelle_2_AXLE_11_819Z_DFZ01}
	\includegraphics[width = 0.49\textwidth]{../AiyuDocs/new_signalplots/Schwelle_2_AXLE_12_819Z_DFZ01}
	
	\includegraphics[width = 0.49\textwidth]{../AiyuDocs/new_signalplots/Schwelle_2_AXLE_41_819Z_DFZ01}
	\includegraphics[width = 0.49\textwidth]{../AiyuDocs/new_signalplots/Schwelle_2_AXLE_42_819Z_DFZ01}
	\caption{Schwelle: \textit{Beschadigte Betonschwellen} (A-118152)}
\end{figure}

\begin{figure}[H]
	\centering
	\includegraphics[width = 0.49\textwidth]{../AiyuDocs/new_signalplots/Fahrbahn_0_AXLE_11_819Z_DFZ01}
	\includegraphics[width = 0.49\textwidth]{../AiyuDocs/new_signalplots/Fahrbahn_0_AXLE_12_819Z_DFZ01}
	
	\includegraphics[width = 0.49\textwidth]{../AiyuDocs/new_signalplots/Fahrbahn_0_AXLE_41_819Z_DFZ01}
	\includegraphics[width = 0.49\textwidth]{../AiyuDocs/new_signalplots/Fahrbahn_0_AXLE_42_819Z_DFZ01}
	\caption{Fahrbahn: \textit{Verletzungen} (A-233599)}
\end{figure}

\begin{figure}[H]
	\centering
	\includegraphics[width = 0.49\textwidth]{../AiyuDocs/new_signalplots/Vignolschiene_12_AXLE_11_819Z_DFZ01}
	\includegraphics[width = 0.49\textwidth]{../AiyuDocs/new_signalplots/Vignolschiene_12_AXLE_12_819Z_DFZ01}
	
	\includegraphics[width = 0.49\textwidth]{../AiyuDocs/new_signalplots/Vignolschiene_12_AXLE_41_819Z_DFZ01}
	\includegraphics[width = 0.49\textwidth]{../AiyuDocs/new_signalplots/Vignolschiene_12_AXLE_42_819Z_DFZ01}
	\caption{Vignolschiene: \textit{Verletzungen} (A-231400)}
\end{figure}

%range
\begin{figure}[H]
	\centering
	\includegraphics[width = 0.49\textwidth]{../AiyuDocs/new_signalplots/Gleis_0_AXLE_11_819Z_DFZ01}
	\includegraphics[width = 0.49\textwidth]{../AiyuDocs/new_signalplots/Gleis_0_AXLE_12_819Z_DFZ01}
	
	\includegraphics[width = 0.49\textwidth]{../AiyuDocs/new_signalplots/Gleis_0_AXLE_41_819Z_DFZ01}
	\includegraphics[width = 0.49\textwidth]{../AiyuDocs/new_signalplots/Gleis_0_AXLE_42_819Z_DFZ01}
	\caption{Gleis-149.8m: \textit{Head-Checking: Periodische Rissbildung / Ausbrechungen an der Fahrkante} (A-184063)}
\end{figure}

\begin{figure}[H]
	\centering
	\includegraphics[width = 0.49\textwidth]{../AiyuDocs/new_signalplots/Schienenzwischenlage_2_AXLE_11_819Z_DFZ01}
	\includegraphics[width = 0.49\textwidth]{../AiyuDocs/new_signalplots/Schienenzwischenlage_2_AXLE_12_819Z_DFZ01}
	
	\includegraphics[width = 0.49\textwidth]{../AiyuDocs/new_signalplots/Schienenzwischenlage_2_AXLE_41_819Z_DFZ01}
	\includegraphics[width = 0.49\textwidth]{../AiyuDocs/new_signalplots/Schienenzwischenlage_2_AXLE_42_819Z_DFZ01}
	\caption{Schienenzwischenlage-5.0m: \textit{Weiterer Abweichungstyp} (A-146358)}
\end{figure}

\begin{figure}[H]
	\centering
	\includegraphics[width = 0.49\textwidth]{../AiyuDocs/new_signalplots/Bankett_1_AXLE_11_819Z_DFZ01}
	\includegraphics[width = 0.49\textwidth]{../AiyuDocs/new_signalplots/Bankett_1_AXLE_12_819Z_DFZ01}
	
	\includegraphics[width = 0.49\textwidth]{../AiyuDocs/new_signalplots/Bankett_1_AXLE_41_819Z_DFZ01}
	\includegraphics[width = 0.49\textwidth]{../AiyuDocs/new_signalplots/Bankett_1_AXLE_42_819Z_DFZ01}
	\caption{Bankett-1169.5m: \textit{Ungenugendes Schotterprofil} (A-103213)}
\end{figure}

%%%%%%%%%%%%%%%%%%%%%%%%%%%%%%%%%%%%%%%%%%%%%%%%%%%%%%%%%%%%%%%%%%%%%%%%%%%%%%%%%%%%%%%%%%%%%%%%
%%% Appendix Implementation
%%%%%%%%%%%%%%%%%%%%%%%%%%%%%%%%%%%%%%%%%%%%%%%%%%%%%%%%%%%%%%%%%%%%%%%%%%%%%%%%%%%%%%%%%%%%%%%
\section{Implementation}
%%%%% Switches
%\begin{comment}
\subsection{Switches}
\begin{figure}[H]
	\centering
	\includegraphics[width = 0.49\textwidth]{../AiyuDocs/new_signalplots/switches_13_AXLE_11_425Z_DFZ01}
	\includegraphics[width = 0.49\textwidth]{../AiyuDocs/new_signalplots/switches_13_AXLE_12_425Z_DFZ01}
	
	\includegraphics[width = 0.49\textwidth]{../AiyuDocs/new_signalplots/switches_13_AXLE_41_425Z_DFZ01}
	\includegraphics[width = 0.49\textwidth]{../AiyuDocs/new_signalplots/switches_13_AXLE_42_425Z_DFZ01}
\end{figure}

\begin{figure}[H]
	\centering
	\includegraphics[width = 0.49\textwidth]{../AiyuDocs/new_signalplots/switches_16_AXLE_11_077Z_DFZ01}
	\includegraphics[width = 0.49\textwidth]{../AiyuDocs/new_signalplots/switches_16_AXLE_12_077Z_DFZ01}
	
	\includegraphics[width = 0.49\textwidth]{../AiyuDocs/new_signalplots/switches_16_AXLE_41_077Z_DFZ01}
	\includegraphics[width = 0.49\textwidth]{../AiyuDocs/new_signalplots/switches_16_AXLE_42_077Z_DFZ01}
\end{figure}

\begin{figure}[H]
	\centering
	\includegraphics[width = 0.49\textwidth]{../AiyuDocs/new_signalplots/switches_17_AXLE_11_425Z_DFZ01}
	\includegraphics[width = 0.49\textwidth]{../AiyuDocs/new_signalplots/switches_17_AXLE_12_425Z_DFZ01}
	
	\includegraphics[width = 0.49\textwidth]{../AiyuDocs/new_signalplots/switches_17_AXLE_41_425Z_DFZ01}
	\includegraphics[width = 0.49\textwidth]{../AiyuDocs/new_signalplots/switches_17_AXLE_42_425Z_DFZ01}
\end{figure}

\begin{figure}[H]
	\centering
	\includegraphics[width = 0.49\textwidth]{../AiyuDocs/new_signalplots/switches_18_AXLE_11_077Z_DFZ01}
	\includegraphics[width = 0.49\textwidth]{../AiyuDocs/new_signalplots/switches_18_AXLE_12_077Z_DFZ01}
	
	\includegraphics[width = 0.49\textwidth]{../AiyuDocs/new_signalplots/switches_18_AXLE_41_077Z_DFZ01}
	\includegraphics[width = 0.49\textwidth]{../AiyuDocs/new_signalplots/switches_18_AXLE_42_077Z_DFZ01}
\end{figure}

%%%%% Ins joints
\subsection{Insulation Joints}
\begin{figure}[H]
	\centering
	\includegraphics[width = 0.49\textwidth]{../AiyuDocs/new_signalplots/insulationjoint_3_AXLE_11_077Z_DFZ01}
	\includegraphics[width = 0.49\textwidth]{../AiyuDocs/new_signalplots/insulationjoint_3_AXLE_12_077Z_DFZ01}
	
	\includegraphics[width = 0.49\textwidth]{../AiyuDocs/new_signalplots/insulationjoint_3_AXLE_41_077Z_DFZ01}
	\includegraphics[width = 0.49\textwidth]{../AiyuDocs/new_signalplots/insulationjoint_3_AXLE_42_077Z_DFZ01}
\end{figure}

\begin{figure}[H]
	\centering
	\includegraphics[width = 0.49\textwidth]{../AiyuDocs/new_signalplots/insulationjoint_5_AXLE_11_077Z_DFZ01}
	\includegraphics[width = 0.49\textwidth]{../AiyuDocs/new_signalplots/insulationjoint_5_AXLE_12_077Z_DFZ01}
	
	\includegraphics[width = 0.49\textwidth]{../AiyuDocs/new_signalplots/insulationjoint_5_AXLE_41_077Z_DFZ01}
	\includegraphics[width = 0.49\textwidth]{../AiyuDocs/new_signalplots/insulationjoint_5_AXLE_42_077Z_DFZ01}
\end{figure}

\begin{figure}[H]
	\centering
	\includegraphics[width = 0.49\textwidth]{../AiyuDocs/new_signalplots/insulationjoint_14_AXLE_11_819Z_DFZ01}
	\includegraphics[width = 0.49\textwidth]{../AiyuDocs/new_signalplots/insulationjoint_14_AXLE_12_819Z_DFZ01}
	
	\includegraphics[width = 0.49\textwidth]{../AiyuDocs/new_signalplots/insulationjoint_14_AXLE_41_819Z_DFZ01}
	\includegraphics[width = 0.49\textwidth]{../AiyuDocs/new_signalplots/insulationjoint_14_AXLE_42_819Z_DFZ01}
\end{figure}

\begin{figure}[H]
	\centering
	\includegraphics[width = 0.49\textwidth]{../AiyuDocs/new_signalplots/insulationjoint_5_AXLE_11_330Z_DFZ01}
	\includegraphics[width = 0.49\textwidth]{../AiyuDocs/new_signalplots/insulationjoint_5_AXLE_12_330Z_DFZ01}
	
	\includegraphics[width = 0.49\textwidth]{../AiyuDocs/new_signalplots/insulationjoint_5_AXLE_41_330Z_DFZ01}
	\includegraphics[width = 0.49\textwidth]{../AiyuDocs/new_signalplots/insulationjoint_5_AXLE_42_330Z_DFZ01}
\end{figure}

%%%%% Defects
\subsection{Defects}
\begin{figure}[H]
	\centering
	\includegraphics[width = 0.49\textwidth]{../AiyuDocs/new_signalplots/Bankett_0_AXLE_11_330Z_DFZ01}
	\includegraphics[width = 0.49\textwidth]{../AiyuDocs/new_signalplots/Bankett_0_AXLE_12_330Z_DFZ01}
	
	\includegraphics[width = 0.49\textwidth]{../AiyuDocs/new_signalplots/Bankett_0_AXLE_41_330Z_DFZ01}
	\includegraphics[width = 0.49\textwidth]{../AiyuDocs/new_signalplots/Bankett_0_AXLE_42_330Z_DFZ01}
	\caption{Bankett}
\end{figure}

\begin{figure}[H]
	\centering
	\includegraphics[width = 0.49\textwidth]{../AiyuDocs/new_signalplots/Befestigung_0_AXLE_11_819Z_DFZ01}
	\includegraphics[width = 0.49\textwidth]{../AiyuDocs/new_signalplots/Befestigung_0_AXLE_12_819Z_DFZ01}
	
	\includegraphics[width = 0.49\textwidth]{../AiyuDocs/new_signalplots/Befestigung_0_AXLE_41_819Z_DFZ01}
	\includegraphics[width = 0.49\textwidth]{../AiyuDocs/new_signalplots/Befestigung_0_AXLE_42_819Z_DFZ01}
	\caption{Befestigung}
\end{figure}

\begin{figure}[H]
	\centering
	\includegraphics[width = 0.49\textwidth]{../AiyuDocs/new_signalplots/Dienstweg_0_AXLE_11_330Z_DFZ01}
	\includegraphics[width = 0.49\textwidth]{../AiyuDocs/new_signalplots/Dienstweg_0_AXLE_12_330Z_DFZ01}
	
	\includegraphics[width = 0.49\textwidth]{../AiyuDocs/new_signalplots/Dienstweg_0_AXLE_41_330Z_DFZ01}
	\includegraphics[width = 0.49\textwidth]{../AiyuDocs/new_signalplots/Dienstweg_0_AXLE_42_330Z_DFZ01}
	\caption{Dienstweg}
\end{figure}

\begin{figure}[H]
	\centering
	\includegraphics[width = 0.49\textwidth]{../AiyuDocs/new_signalplots/Entwässerungsgraben_0_AXLE_11_077Z_DFZ01}
	\includegraphics[width = 0.49\textwidth]{../AiyuDocs/new_signalplots/Entwässerungsgraben_0_AXLE_12_077Z_DFZ01}
	
	\includegraphics[width = 0.49\textwidth]{../AiyuDocs/new_signalplots/Entwässerungsgraben_0_AXLE_41_077Z_DFZ01}
	\includegraphics[width = 0.49\textwidth]{../AiyuDocs/new_signalplots/Entwässerungsgraben_0_AXLE_42_077Z_DFZ01}
	\caption{Entwässerungsgraben}
\end{figure}

\begin{figure}[H]
	\centering
	\includegraphics[width = 0.49\textwidth]{../AiyuDocs/new_signalplots/EKW_DKW_1_AXLE_11_425Z_DFZ01}
	\includegraphics[width = 0.49\textwidth]{../AiyuDocs/new_signalplots/EKW_DKW_1_AXLE_12_425Z_DFZ01}
	
	\includegraphics[width = 0.49\textwidth]{../AiyuDocs/new_signalplots/EKW_DKW_1_AXLE_41_425Z_DFZ01}
	\includegraphics[width = 0.49\textwidth]{../AiyuDocs/new_signalplots/EKW_DKW_1_AXLE_42_425Z_DFZ01}
	\caption{EKW / DKW}
\end{figure}

\begin{figure}[H]
	\centering
	\includegraphics[width = 0.49\textwidth]{../AiyuDocs/new_signalplots/Fahrbahn_0_AXLE_11_330Z_DFZ01}
	\includegraphics[width = 0.49\textwidth]{../AiyuDocs/new_signalplots/Fahrbahn_0_AXLE_12_330Z_DFZ01}
	
	\includegraphics[width = 0.49\textwidth]{../AiyuDocs/new_signalplots/Fahrbahn_0_AXLE_41_330Z_DFZ01}
	\includegraphics[width = 0.49\textwidth]{../AiyuDocs/new_signalplots/Fahrbahn_0_AXLE_42_330Z_DFZ01}
	\caption{Fahrbahn}
\end{figure}

\begin{figure}[H]
	\centering
	\includegraphics[width = 0.49\textwidth]{../AiyuDocs/new_signalplots/Gleis_3_AXLE_11_330Z_DFZ01}
	\includegraphics[width = 0.49\textwidth]{../AiyuDocs/new_signalplots/Gleis_3_AXLE_12_330Z_DFZ01}
	
	\includegraphics[width = 0.49\textwidth]{../AiyuDocs/new_signalplots/Gleis_3_AXLE_41_330Z_DFZ01}
	\includegraphics[width = 0.49\textwidth]{../AiyuDocs/new_signalplots/Gleis_3_AXLE_42_330Z_DFZ01}
	\caption{Gleis}
\end{figure}

\begin{figure}[H]
	\centering
	\includegraphics[width = 0.49\textwidth]{../AiyuDocs/new_signalplots/Gleisbettung_0_AXLE_11_330Z_DFZ01}
	\includegraphics[width = 0.49\textwidth]{../AiyuDocs/new_signalplots/Gleisbettung_0_AXLE_12_330Z_DFZ01}
	
	\includegraphics[width = 0.49\textwidth]{../AiyuDocs/new_signalplots/Gleisbettung_0_AXLE_41_330Z_DFZ01}
	\includegraphics[width = 0.49\textwidth]{../AiyuDocs/new_signalplots/Gleisbettung_0_AXLE_42_330Z_DFZ01}
	\caption{Gleisbettung}
\end{figure}

\begin{figure}[H]
	\centering
	\includegraphics[width = 0.49\textwidth]{../AiyuDocs/new_signalplots/Herzstück_1_AXLE_11_077Z_DFZ01}
	\includegraphics[width = 0.49\textwidth]{../AiyuDocs/new_signalplots/Herzstück_1_AXLE_12_077Z_DFZ01}
	
	\includegraphics[width = 0.49\textwidth]{../AiyuDocs/new_signalplots/Herzstück_1_AXLE_41_077Z_DFZ01}
	\includegraphics[width = 0.49\textwidth]{../AiyuDocs/new_signalplots/Herzstück_1_AXLE_42_077Z_DFZ01}
	\caption{Herzstück}
\end{figure}

\begin{figure}[H]
	\centering
	\includegraphics[width = 0.49\textwidth]{../AiyuDocs/new_signalplots/Schiene_16_AXLE_11_077Z_DFZ01}
	\includegraphics[width = 0.49\textwidth]{../AiyuDocs/new_signalplots/Schiene_16_AXLE_12_077Z_DFZ01}
	
	\includegraphics[width = 0.49\textwidth]{../AiyuDocs/new_signalplots/Schiene_16_AXLE_41_077Z_DFZ01}
	\includegraphics[width = 0.49\textwidth]{../AiyuDocs/new_signalplots/Schiene_16_AXLE_42_077Z_DFZ01}
	\caption{Schiene}
\end{figure}

\begin{figure}[H]
	\centering
	\includegraphics[width = 0.49\textwidth]{../AiyuDocs/new_signalplots/Schienenzwischenlage_1_AXLE_11_819Z_DFZ01}
	\includegraphics[width = 0.49\textwidth]{../AiyuDocs/new_signalplots/Schienenzwischenlage_1_AXLE_12_819Z_DFZ01}
	
	\includegraphics[width = 0.49\textwidth]{../AiyuDocs/new_signalplots/Schienenzwischenlage_1_AXLE_41_819Z_DFZ01}
	\includegraphics[width = 0.49\textwidth]{../AiyuDocs/new_signalplots/Schienenzwischenlage_1_AXLE_42_819Z_DFZ01}
	\caption{Schienenzwischenlage}
\end{figure}

\begin{figure}[H]
	\centering
	\includegraphics[width = 0.49\textwidth]{../AiyuDocs/new_signalplots/Schotter_2_AXLE_11_077Z_DFZ01}
	\includegraphics[width = 0.49\textwidth]{../AiyuDocs/new_signalplots/Schotter_2_AXLE_12_077Z_DFZ01}
	
	\includegraphics[width = 0.49\textwidth]{../AiyuDocs/new_signalplots/Schotter_2_AXLE_41_077Z_DFZ01}
	\includegraphics[width = 0.49\textwidth]{../AiyuDocs/new_signalplots/Schotter_2_AXLE_42_077Z_DFZ01}
	\caption{Schotter}
\end{figure}

\begin{figure}[H]
	\centering
	\includegraphics[width = 0.49\textwidth]{../AiyuDocs/new_signalplots/Schweissverbindung_0_AXLE_11_819Z_DFZ01}
	\includegraphics[width = 0.49\textwidth]{../AiyuDocs/new_signalplots/Schweissverbindung_0_AXLE_12_819Z_DFZ01}
	
	\includegraphics[width = 0.49\textwidth]{../AiyuDocs/new_signalplots/Schweissverbindung_0_AXLE_41_819Z_DFZ01}
	\includegraphics[width = 0.49\textwidth]{../AiyuDocs/new_signalplots/Schweissverbindung_0_AXLE_42_819Z_DFZ01}
	\caption{Schweissverbindung}
\end{figure}

\begin{figure}[H]
	\centering
	\includegraphics[width = 0.49\textwidth]{../AiyuDocs/new_signalplots/Schwelle_0_AXLE_11_077Z_DFZ01}
	\includegraphics[width = 0.49\textwidth]{../AiyuDocs/new_signalplots/Schwelle_0_AXLE_12_077Z_DFZ01}
	
	\includegraphics[width = 0.49\textwidth]{../AiyuDocs/new_signalplots/Schwelle_0_AXLE_41_077Z_DFZ01}
	\includegraphics[width = 0.49\textwidth]{../AiyuDocs/new_signalplots/Schwelle_0_AXLE_42_077Z_DFZ01}
	\caption{Schwelle}
\end{figure}

\begin{figure}[H]
	\centering
	\includegraphics[width = 0.49\textwidth]{../AiyuDocs/new_signalplots/Stockschiene_1_AXLE_11_819Z_DFZ01}
	\includegraphics[width = 0.49\textwidth]{../AiyuDocs/new_signalplots/Stockschiene_1_AXLE_12_819Z_DFZ01}
	
	\includegraphics[width = 0.49\textwidth]{../AiyuDocs/new_signalplots/Stockschiene_1_AXLE_41_819Z_DFZ01}
	\includegraphics[width = 0.49\textwidth]{../AiyuDocs/new_signalplots/Stockschiene_1_AXLE_42_819Z_DFZ01}
	\caption{Stockschiene}
\end{figure}

\begin{figure}[H]
	\centering
	\includegraphics[width = 0.49\textwidth]{../AiyuDocs/new_signalplots/Vignolschiene_13_AXLE_11_819Z_DFZ01}
	\includegraphics[width = 0.49\textwidth]{../AiyuDocs/new_signalplots/Vignolschiene_13_AXLE_12_819Z_DFZ01}
	
	\includegraphics[width = 0.49\textwidth]{../AiyuDocs/new_signalplots/Vignolschiene_13_AXLE_41_819Z_DFZ01}
	\includegraphics[width = 0.49\textwidth]{../AiyuDocs/new_signalplots/Vignolschiene_13_AXLE_42_819Z_DFZ01}
	\caption{Vignolschiene}
\end{figure}

\begin{figure}[H]
	\centering
	\includegraphics[width = 0.49\textwidth]{../AiyuDocs/new_signalplots/Weiche_0_AXLE_11_819Z_DFZ01}
	\includegraphics[width = 0.49\textwidth]{../AiyuDocs/new_signalplots/Weiche_0_AXLE_12_819Z_DFZ01}
	
	\includegraphics[width = 0.49\textwidth]{../AiyuDocs/new_signalplots/Weiche_0_AXLE_41_819Z_DFZ01}
	\includegraphics[width = 0.49\textwidth]{../AiyuDocs/new_signalplots/Weiche_0_AXLE_42_819Z_DFZ01}
	\caption{Vignolschiene}
\end{figure}

\begin{figure}[H]
	\centering
	\includegraphics[width = 0.49\textwidth]{../AiyuDocs/new_signalplots/Zungenschiene_0_AXLE_11_425Z_DFZ01}
	\includegraphics[width = 0.49\textwidth]{../AiyuDocs/new_signalplots/Zungenschiene_0_AXLE_12_425Z_DFZ01}
	
	\includegraphics[width = 0.49\textwidth]{../AiyuDocs/new_signalplots/Zungenschiene_0_AXLE_41_425Z_DFZ01}
	\includegraphics[width = 0.49\textwidth]{../AiyuDocs/new_signalplots/Zungenschiene_0_AXLE_42_425Z_DFZ01}
	\caption{Zungenschiene}
\end{figure}
%\end{comment}

%%%%%%%%%%%%%%%%%%%%%%%%%%%%%%%%%%%%%%%%%%%%%%%%%%%%%%%%%%%%%%%%%%%%%%%%%%%%%%%%%%%%%%%%%%%%%%%%
%%% Appendix Evaluation
%%%%%%%%%%%%%%%%%%%%%%%%%%%%%%%%%%%%%%%%%%%%%%%%%%%%%%%%%%%%%%%%%%%%%%%%%%%%%%%%%%%%%%%%%%%%%%%
\chapter{Evaluation}


\section{All 4 axle channels}
\begin{table}[H]
	\centering
\begin{tabular}{|l|cccc|}
	\hline
	\backslashbox{run \#}{Metrics}
	&{L}&{ACC}&{TP}&{TN} \\\hline\hline
		1 & $-$ & $-$ & $-$ & $-$ \\ \hline
		2 & $-$ & $-$ & $-$ & $-$ \\ \hline
		3 & $-$ & $-$ & $-$ & $-$ \\ \hline
		4 & $-$ & $-$ & $-$ & $-$ \\ \hline
		5 & $-$ & $-$ & $-$ & $-$ \\ \hline
		6 & $-$ & $-$ & $-$ & $-$ \\ \hline
		7 & $-$ & $-$ & $-$ & $-$ \\ \hline
		8 & $-$ & $-$ & $-$ & $-$ \\ \hline
		9 & $-$ & $-$ & $-$ & $-$ \\ \hline
		10 & $-$ & $-$ & $-$ & $-$ \\ \hline
\end{tabular}
	\caption{Experiment result of 10 runs}
\end{table}
confusion matrix, epoch plots for each experiment

\section{Single axle channel}
\begin{table}[H]
	\centering
\begin{tabular}{|l|cccc|}
	\hline
	\backslashbox{run \#}{Metrics}
	&{L}&{ACC}&{TP}&{TN} \\\hline\hline
		1 & $-$ & $-$ & $-$ & $-$ \\ \hline
		2 & $-$ & $-$ & $-$ & $-$ \\ \hline
		3 & $-$ & $-$ & $-$ & $-$ \\ \hline
		4 & $-$ & $-$ & $-$ & $-$ \\ \hline
		5 & $-$ & $-$ & $-$ & $-$ \\ \hline
		6 & $-$ & $-$ & $-$ & $-$ \\ \hline
		7 & $-$ & $-$ & $-$ & $-$ \\ \hline
		8 & $-$ & $-$ & $-$ & $-$ \\ \hline
		9 & $-$ & $-$ & $-$ & $-$ \\ \hline
		10 & $-$ & $-$ & $-$ & $-$ \\ \hline
\end{tabular}
	\caption{Experiment result of 10 runs}
\end{table}
confusion matrix, epoch plots for each experiment
\section{Low pass-filtered single axle channel}
\begin{table}[H]
	\centering
\begin{tabular}{|l|cccc|}
	\hline
	\backslashbox{run \#}{Metrics}
	&{L}&{ACC}&{TP}&{TN} \\\hline\hline
		1 & $-$ & $-$ & $-$ & $-$ \\ \hline
		2 & $-$ & $-$ & $-$ & $-$ \\ \hline
		3 & $-$ & $-$ & $-$ & $-$ \\ \hline
		4 & $-$ & $-$ & $-$ & $-$ \\ \hline
		5 & $-$ & $-$ & $-$ & $-$ \\ \hline
		6 & $-$ & $-$ & $-$ & $-$ \\ \hline
		7 & $-$ & $-$ & $-$ & $-$ \\ \hline
		8 & $-$ & $-$ & $-$ & $-$ \\ \hline
		9 & $-$ & $-$ & $-$ & $-$ \\ \hline
		10 & $-$ & $-$ & $-$ & $-$ \\ \hline
\end{tabular}
	\caption{Experiment result of 10 runs}
\end{table}
confusion matrix, epoch plots for each experiment
\section{Single axle channel with speed}
\begin{table}[H]
	\centering
\begin{tabular}{|l|cccc|}
	\hline
	\backslashbox{run \#}{Metrics}
	&{L}&{ACC}&{TP}&{TN} \\\hline\hline
		1 & $-$ & $-$ & $-$ & $-$ \\ \hline
		2 & $-$ & $-$ & $-$ & $-$ \\ \hline
		3 & $-$ & $-$ & $-$ & $-$ \\ \hline
		4 & $-$ & $-$ & $-$ & $-$ \\ \hline
		5 & $-$ & $-$ & $-$ & $-$ \\ \hline
		6 & $-$ & $-$ & $-$ & $-$ \\ \hline
		7 & $-$ & $-$ & $-$ & $-$ \\ \hline
		8 & $-$ & $-$ & $-$ & $-$ \\ \hline
		9 & $-$ & $-$ & $-$ & $-$ \\ \hline
		10 & $-$ & $-$ & $-$ & $-$ \\ \hline
\end{tabular}
	\caption{Experiment result of 10 runs}
\end{table}
confusion matrix, epoch plots for each experiment
\section{Lowpass filtered single axle channel with speed}
\begin{table}[H]
	\centering
\begin{tabular}{|l|cccc|}
	\hline
	\backslashbox{run \#}{Metrics}
	&{L}&{ACC}&{TP}&{TN} \\\hline\hline
		1 & $-$ & $-$ & $-$ & $-$ \\ \hline
		2 & $-$ & $-$ & $-$ & $-$ \\ \hline
		3 & $-$ & $-$ & $-$ & $-$ \\ \hline
		4 & $-$ & $-$ & $-$ & $-$ \\ \hline
		5 & $-$ & $-$ & $-$ & $-$ \\ \hline
		6 & $-$ & $-$ & $-$ & $-$ \\ \hline
		7 & $-$ & $-$ & $-$ & $-$ \\ \hline
		8 & $-$ & $-$ & $-$ & $-$ \\ \hline
		9 & $-$ & $-$ & $-$ & $-$ \\ \hline
		10 & $-$ & $-$ & $-$ & $-$ \\ \hline
\end{tabular}
	\caption{Experiment result of 10 runs}
\end{table}
confusion matrix, epoch plots for each experiment



%\foreach \n in {0,1,2,3,4,5,6,7,9}{
%	\begin{figure}[H]
%	\foreach \k in {0,1,2,3,4,5,6,7,8,9}{
%		\ifnum\n=\k
%		\else
%		\begin{subfigure}[b]{0.19\textwidth}
%			\centering
%			\includegraphics[width=\textwidth]{../Cleaning/wekapics/WEKAmnist_all2000CLASS\n_\k}
%			\caption{Class \n vs\k}
%			\label{class_\n\k}
%		\end{subfigure}
%		\fi
%		}
%	\caption{Comparisons with Class \n}
%	\end{figure}
%	\ifoddpage
%		\clearpage
%	\else
%	\fi
%}



\end{document}
