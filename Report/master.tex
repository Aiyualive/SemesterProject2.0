%\documentclass[11pt]{report}
%\usepackage[a4paper]{geometry}

\documentclass[ % the name of the author
                    author={Aiyu Liu},
                % the name of the supervisor
                supervisor={Supervised by: Cyprien Hoelzl, Prof. Eleni Chatzi},
                % the degree programme
                    degree={BSc},
                % the dissertation    title (which cannot be blank)
                     title={Semester Project Report},
                % the dissertation subtitle (which can    be blank)
                  subtitle={Data-driven identificiation and classification of rail surface defectse},
                % the dissertation     type
                %  type={enterprise},
                % the year of submission
                      year={2019} ]{dissertation}
%
\documentclass[ % the name of the author
                    author={Aiyu Liu},
                % the name of the supervisor
                supervisor={Supervised by: Cyprien Hoelzl, Prof. Eleni Chatzi},
                % the degree programme
                    degree={BSc},
                % the dissertation    title (which cannot be blank)
                     title={Semester Project Report},
                % the dissertation subtitle (which can    be blank)
                  subtitle={Data-driven identificiation and classification of rail surface defectse},
                % the dissertation     type
                %  type={enterprise},
                % the year of submission
                      year={2019} ]{dissertation}
%
\documentclass[ % the name of the author
                    author={Aiyu Liu},
                % the name of the supervisor
                supervisor={Supervised by: Cyprien Hoelzl, Prof. Eleni Chatzi},
                % the degree programme
                    degree={BSc},
                % the dissertation    title (which cannot be blank)
                     title={Semester Project Report},
                % the dissertation subtitle (which can    be blank)
                  subtitle={Data-driven identificiation and classification of rail surface defectse},
                % the dissertation     type
                %  type={enterprise},
                % the year of submission
                      year={2019} ]{dissertation}
%\input{config/prelude}



%%% URL line break      
\expandafter\def\expandafter\UrlBreaks\expandafter{\UrlBreaks%  save the current one
  \do\a\do\b\do\c\do\d\do\e\do\f\do\g\do\h\do\i\do\j%
  \do\k\do\l\do\m\do\n\do\o\do\p\do\q\do\r\do\s\do\t%
  \do\u\do\v\do\w\do\x\do\y\do\z\do\A\do\B\do\C\do\D%
  \do\E\do\F\do\G\do\H\do\I\do\J\do\K\do\L\do\M\do\N%
  \do\O\do\P\do\Q\do\R\do\S\do\T\do\U\do\V\do\W\do\X%
  \do\Y\do\Z}
\usepackage{minted}

\usepackage{caption}
\usepackage{amsmath}
\usepackage{todonotes}
\setlength\marginparwidth{0.75in}
\usepackage{amssymb}
%\PassOptionsToPackage{hyphens}{url}
%\usepackage[hidelinks]{hyperref}
\usepackage{diagbox}
\usepackage{algorithm}
\usepackage{graphicx}
\usepackage{subcaption}
\usepackage{makecell}
\usepackage{comment}
\usepackage[noend]{algpseudocode}
%\usepackage[usenames, dvipsnames, svgnames, table]{xcolor}
\usepackage{bm}
\usepackage{array}
\usepackage{pgffor}
\usepackage{changepage}
\usepackage{pbox}
\usepackage{tabularx}
\usepackage{afterpage}
\usepackage{forest} % for making directory trees
\usepackage{emptypage} % get rid of numbering after title page
\usepackage{enumitem}
\usepackage{filecontents}

\usepackage{pgfplotstable} 
\usepackage{booktabs} 
\usepackage{filecontents}

%\usepackage[lite]{mtpro2}
%\newcommand{\todo}[1]{{\color{red} #1}}

\newcommand{\forceindent}{\leavevmode{\parindent=1cm\indent}}
\newcommand{\forceindenthalf}{\leavevmode{\parindent=0.5cm\indent}}

\definecolor{dkgreen}{rgb}{0,0.6,0}
\definecolor{gray}{rgb}{0.5,0.5,0.5}
\definecolor{mauve}{rgb}{0.58,0,0.82}
\lstset{%frame=tb,
%  language=Java,
  aboveskip=3mm,
  belowskip=3mm,
  showstringspaces=false,
  columns=flexible,
  basicstyle={\small\ttfamily},
  numbers=none,
  numberstyle=\tiny\color{gray},
  keywordstyle=\color{blue},
  commentstyle=\color{dkgreen},
  stringstyle=\color{mauve},
  breaklines=true,
  breakatwhitespace=true,
  tabsize=2
}
%backgroundcolor = \color{lightgray}

\newcommand{\specialcell}[2][c]{%
  \begin{tabular}[#1]{@{}c@{}}#2\end{tabular}}

\setlength\arrayrulewidth{1pt} %Change the line thickness of latex tables




%%% URL line break      
\expandafter\def\expandafter\UrlBreaks\expandafter{\UrlBreaks%  save the current one
  \do\a\do\b\do\c\do\d\do\e\do\f\do\g\do\h\do\i\do\j%
  \do\k\do\l\do\m\do\n\do\o\do\p\do\q\do\r\do\s\do\t%
  \do\u\do\v\do\w\do\x\do\y\do\z\do\A\do\B\do\C\do\D%
  \do\E\do\F\do\G\do\H\do\I\do\J\do\K\do\L\do\M\do\N%
  \do\O\do\P\do\Q\do\R\do\S\do\T\do\U\do\V\do\W\do\X%
  \do\Y\do\Z}
\usepackage{minted}

\usepackage{caption}
\usepackage{amsmath}
\usepackage{todonotes}
\setlength\marginparwidth{0.75in}
\usepackage{amssymb}
%\PassOptionsToPackage{hyphens}{url}
%\usepackage[hidelinks]{hyperref}
\usepackage{diagbox}
\usepackage{algorithm}
\usepackage{graphicx}
\usepackage{subcaption}
\usepackage{makecell}
\usepackage{comment}
\usepackage[noend]{algpseudocode}
%\usepackage[usenames, dvipsnames, svgnames, table]{xcolor}
\usepackage{bm}
\usepackage{array}
\usepackage{pgffor}
\usepackage{changepage}
\usepackage{pbox}
\usepackage{tabularx}
\usepackage{afterpage}
\usepackage{forest} % for making directory trees
\usepackage{emptypage} % get rid of numbering after title page
\usepackage{enumitem}
\usepackage{filecontents}

\usepackage{pgfplotstable} 
\usepackage{booktabs} 
\usepackage{filecontents}

%\usepackage[lite]{mtpro2}
%\newcommand{\todo}[1]{{\color{red} #1}}

\newcommand{\forceindent}{\leavevmode{\parindent=1cm\indent}}
\newcommand{\forceindenthalf}{\leavevmode{\parindent=0.5cm\indent}}

\definecolor{dkgreen}{rgb}{0,0.6,0}
\definecolor{gray}{rgb}{0.5,0.5,0.5}
\definecolor{mauve}{rgb}{0.58,0,0.82}
\lstset{%frame=tb,
%  language=Java,
  aboveskip=3mm,
  belowskip=3mm,
  showstringspaces=false,
  columns=flexible,
  basicstyle={\small\ttfamily},
  numbers=none,
  numberstyle=\tiny\color{gray},
  keywordstyle=\color{blue},
  commentstyle=\color{dkgreen},
  stringstyle=\color{mauve},
  breaklines=true,
  breakatwhitespace=true,
  tabsize=2
}
%backgroundcolor = \color{lightgray}

\newcommand{\specialcell}[2][c]{%
  \begin{tabular}[#1]{@{}c@{}}#2\end{tabular}}

\setlength\arrayrulewidth{1pt} %Change the line thickness of latex tables




%%% URL line break      
\expandafter\def\expandafter\UrlBreaks\expandafter{\UrlBreaks%  save the current one
  \do\a\do\b\do\c\do\d\do\e\do\f\do\g\do\h\do\i\do\j%
  \do\k\do\l\do\m\do\n\do\o\do\p\do\q\do\r\do\s\do\t%
  \do\u\do\v\do\w\do\x\do\y\do\z\do\A\do\B\do\C\do\D%
  \do\E\do\F\do\G\do\H\do\I\do\J\do\K\do\L\do\M\do\N%
  \do\O\do\P\do\Q\do\R\do\S\do\T\do\U\do\V\do\W\do\X%
  \do\Y\do\Z}
\usepackage{minted}

\usepackage{caption}
\usepackage{amsmath}
\usepackage{todonotes}
\setlength\marginparwidth{0.75in}
\usepackage{amssymb}
%\PassOptionsToPackage{hyphens}{url}
%\usepackage[hidelinks]{hyperref}
\usepackage{diagbox}
\usepackage{algorithm}
\usepackage{graphicx}
\usepackage{subcaption}
\usepackage{makecell}
\usepackage{comment}
\usepackage[noend]{algpseudocode}
%\usepackage[usenames, dvipsnames, svgnames, table]{xcolor}
\usepackage{bm}
\usepackage{array}
\usepackage{pgffor}
\usepackage{changepage}
\usepackage{pbox}
\usepackage{tabularx}
\usepackage{afterpage}
\usepackage{forest} % for making directory trees
\usepackage{emptypage} % get rid of numbering after title page
\usepackage{enumitem}
\usepackage{filecontents}

\usepackage{pgfplotstable} 
\usepackage{booktabs} 
\usepackage{filecontents}

%\usepackage[lite]{mtpro2}
%\newcommand{\todo}[1]{{\color{red} #1}}

\newcommand{\forceindent}{\leavevmode{\parindent=1cm\indent}}
\newcommand{\forceindenthalf}{\leavevmode{\parindent=0.5cm\indent}}

\definecolor{dkgreen}{rgb}{0,0.6,0}
\definecolor{gray}{rgb}{0.5,0.5,0.5}
\definecolor{mauve}{rgb}{0.58,0,0.82}
\lstset{%frame=tb,
%  language=Java,
  aboveskip=3mm,
  belowskip=3mm,
  showstringspaces=false,
  columns=flexible,
  basicstyle={\small\ttfamily},
  numbers=none,
  numberstyle=\tiny\color{gray},
  keywordstyle=\color{blue},
  commentstyle=\color{dkgreen},
  stringstyle=\color{mauve},
  breaklines=true,
  breakatwhitespace=true,
  tabsize=2
}
%backgroundcolor = \color{lightgray}

\newcommand{\specialcell}[2][c]{%
  \begin{tabular}[#1]{@{}c@{}}#2\end{tabular}}

\setlength\arrayrulewidth{1pt} %Change the line thickness of latex tables


%%%%%%%%%%%%%%%%%%
% BEGIN DOCUMENT %
%%%%%%%%%%%%%%%%%%

%\usepackage[final]{listings}

%\input{csvfiles}

\begin{document}

\maketitle
%\frontmatter
%\makedecl
\newpage
\chapter*{Acknowledgements}
This semester project would not be possible without the help of my supervisors, Cyprien Hoelzl and professor Eleni Chatzi. I first reached out to Eleni for a semester project opportunity during ETH week -- for the purpose of improving my skills in doing research. Shortly therafter, I was introduced to her PhD student, Cyprien Hoelzl, about a project revolving around identifying and classiying defects on train tracks.

Cyprien has been an exceptional mentor throughout this entire project. I received all the acdemic guidance necessary and whenever I had issues, I could always drop by his office or text him, after which helpful answers would promptly ensue. From the beginning, I could tell that he is down-to-earth, very intelligent and possess great specialization in the field of train maintenance and monitoring. He is very good at explaining difficult concepts (with his quick and intuitive hand-drawings) and provided me with many informative resources. Furthermore, he truly cared about my progress, goes out of his way to aid me, and provides constructive feedback for everything I present to him. 

Eleni has also been very supportive about my progress, always arranging intermediate update sessions and staying on top of the project. These have been a great driver in keeping me accountable and making further progress. Eleni is very approachable, kind, and great at providing feedback at the intermediate update sessions. She is extremely active in her endeavours and one can that she is an expert in the field of Structural Health Monitoring (among others).

Finally, I greatly appreciate the help that I have received from Eleni's other PhD student, Harry (Mylonas Charilaos). He has provided me with very informative tools/feedback for my work with neural network architectures. Although interactions were few, one can immediately tell that he is very knowledgeable about the field of machine learning.
 
I have great gratitude for this opportunity working alongside Eleni and her multi-talented team. It has been a pleasant and educational experience. I sincerely could not ask for better supervisors.


\newpage



%\include{abstract}
%\mainmatter
%\setcounter{tocdepth}{3}
%\setcounter{secnumdepth}{3} dunno what this does.
{
  \hypersetup{linkcolor=black}
  \tableofcontents
  % Two times compilation
  %\listoffigures
  %\listoftables
   
}
\newpage
%%%%%%%%%%%%%%%%%%%%%%%%%%%%%%%%%%%%%%%%%%%%%%%%%%%%%%%%%%%%%%%%%%%%%%%%%%%%%%%%%%%%%%%%%%%%%%%%
%%% Intro
%%%%%%%%%%%%%%%%%%%%%%%%%%%%%%%%%%%%%%%%%%%%%%%%%%%%%%%%%%%%%%%%%%%%%%%%%%%%%%%%%%%%%%%%%%%%%%%%
\chapter{Introduction}
%%%%%%%%%%% PARSER %%%%%%%%%%%
\section{Problem description and motivation}
Railway companies need to continuously and sufficiently maintain the train tracks and optimally detect defects in order to have a more punctual and more effective train system. However, the current system is expensive, time consuming and ineffective. That is, maintenance agents need to walk along tracks and check them for defects. For visualisation purposes, there is roughly 5200 km of rails in Switzerland which needs to be inspected by 40 experienced inspectors.


In order to cope with this issue, Swiss Federal Railways (SBB) has specifically built two new diagnostic vehicles designed for defect identification among other purposes. For this, two accelerometers have been installed at the front and back of the vehicle to collect the signal responses from the wheel and the train track
\todo{insert picture, mention boogey?}

A defect in train tracks can be seen as a discontinuity. As a train passes over this discontinuity, it will result in a perturbation that can be detected by sensors. It is our main assumption that each type of defect will have a specific signature that will allow its identification and classification. This is similar to the idea presented in \todo{https://blog.goodaudience.com/introduction-to-1d-convolutional-neural-networks-in-keras-for-time-sequences-3a7ff801a2cf}

By succesfully identifying and classifying the defects, we take one step further towards reducing delays and making trains more punctual and reliable. The first step in this process consists of identification and classification, while the second step consists of future defect prediction.

\section{Objective}
The objective 

\section{Defects}
Evidently, a defect can be seen as a deviation from the standard train track. For the exact defect type, SBB has self-constructed a database for the individual defect definitions \todo{is this a recognized system?}. Here is a few examples:


Generally, a defect can be split into two overarching types: line-defects and type-defects. For this project
SBB has done the classification themselves
Defect can be of any type, which defects do we want to focus on
\todo{insert pictures}

\section{Data}
Data is provided by SBB

\section{Code}
The code can be found on github: \url{sdds}


%%%%%%%%%%%%%%%%%%%%%%%%%%%%%%%%%%%%%%%%%%%%%%%%%%%%%%%%%%%%%%%%%%%%%%%%%%%%%%%%%%%%%%%%%%%%%%%%
%%% Design and implementation
%%%%%%%%%%%%%%%%%%%%%%%%%%%%%%%%%%%%%%%%%%%%%%%%%%%%%%%%%%%%%%%%%%%%%%%%%%%%%%%%%%%%%%%%%%%%%%%%
\chapter{Design and Implementation}
First we need to analyse the data, show a few defects and their signals \todo{appendix for more signals?}


\section{Peak windows}
To find,
we can change the parameters for the peak findings
\section{Neural network architecture}

Trained a neural network, although we were only able to achieve max
Based on the analysis we

\section{Visualisation}


%%%%%%%%%%%%%%%%%%%%%%%%%%%%%%%%%%%%%%%%%%%%%%%%%%%%%%%%%%%%%%%%%%%%%%%%%%%%%%%%%%%%%%%%%%%%%%%%
%%% Evaluation
%%%%%%%%%%%%%%%%%%%%%%%%%%%%%%%%%%%%%%%%%%%%%%%%%%%%%%%%%%%%%%%%%%%%%%%%%%%%%%%%%%%%%%%%%%%%%%%%
\chapter{Evaluation}
\section{Results}
\section{Discussion}

%%%%%%%%%%%%%%%%%%%%%%%%%%%%%%%%%%%%%%%%%%%%%%%%%%%%%%%%%%%%%%%%%%%%%%%%%%%%%%%%%%%%%%%%%%%%%%%%
%%% Conclusion and future work
%%%%%%%%%%%%%%%%%%%%%%%%%%%%%%%%%%%%%%%%%%%%%%%%%%%%%%%%%%%%%%%%%%%%%%%%%%%%%%%%%%%%%%%%%%%%%%%%
\chapter{Conclusion and future work}
\section{Conclusion}
\section{Future work}

\begin{itemize}
	\item
\end{itemize}

\begin{itemize}
	\item 
\end{itemize}


\newpage
\section{TODO}
\begin{itemize}
	\item very fast speed, overlap between switch and ins, old vs new rail, ax1 arrow 2 arrow 3 arrow 4
	\item 3D plots?
	\item change the defect library to use pandas instead?
	\item visualise what the network is doing using Harry's code
	\item use speed as a feature also
\end{itemize}


\newpage
\section{Notes}
1D convolution tutorial
Height = acc length
Width = the number of features
Output is determined by kernnel size and height of data

Misc:
\begin{itemize}
	\item \verb|pd.options.display.max_rows = 15|
	\item \verb|#np.bincount(y.class_label.values)/4 where does 151.5 coem from??|
\end{itemize}

whats this
\begin{minted}{python}
def conv(df):
    """
    has to be series
    """
    return np.vstack([v for v in df])

dup_ins = s_features.ins_joints.copy()[['accelerations']]
dup_swi = s_features.switches.copy()[['accelerations']]
dup_def = s_features.defects.copy()[['accelerations']]

dup_ins['accelerations'] = np.sum(conv(dup_ins.accelerations),1)
dup_swi['accelerations'] = np.sum(conv(dup_swi.accelerations),1)
dup_def['accelerations'] = np.sum(conv(dup_def.accelerations),1)

# s_features.ins_joints[['vehicle_speed(m/s)', 'Axle', 'campagin_ID']].duplicated() or

idx_ins = dup_ins.accelerations.duplicated()
idx_swi = dup_swi.accelerations.duplicated()
idx_def = dup_def.accelerations.duplicated()
new_ins = s_features.ins_joints[~idx_ins]
new_swi = s_features.switches[~idx_swi]
new_def = s_features.switches[~idx_def]

print("Duplcated samples: ", len(dup_ins) - len(new_ins))
print("Duplcated samples: ", len(dup_swi) - len(new_swi))
print("Duplcated samples: ", len(dup_def) - len(new_def))

# Load weight example 
# Could just save entire model and then load entire model
# Could also make this into a function
clf2 = NN(N_FEATURES, N_CLASSES)
clf2.prepare_data(X, y)
clf2.make_model2()
clf2.load_weights('model_01-12-2019_150004.hdf5')
clf2.predict() ### on validation set
clf2.measure_performance(accuracy_score)
\end{minted}

Test sample
\begin{minted}{python}
ii = pd.DataFrame([
    [np.array([1,2]),2], 
    [np.array([1,2]),2], 
    [np.array([1,2]),2]])
    
    
x = a
[u,I,J] = unique(x, 'rows', 'first')
hasDuplicates = size(u,1) < size(x,1)
ixDupRows = setdiff(1:size(x,1), I)
dupRowValues = x(ixDupRows,:)

s_features.ins_joints.timestamps[:2].duplicated()
\end{minted}

\chapter{Appendix}
\todo{Figure out references}
\todo{New paper with train}

\newpage
\cleardoublepage
\appendix
\chapter{Appendix}
\label{code}
\inputminted[linenos, fontsize = \scriptsize]{haskell}{../src/utils/defect_utils.py}

\end{document}
